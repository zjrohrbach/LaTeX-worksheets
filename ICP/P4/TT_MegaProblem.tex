\documentclass[12pt]{exam}
\usepackage[phy]{template-for-exam}
\usepackage{tikz,ifthen,multicol,siunitx}
\footer{}{}{}
\header{}{}{}
\shadedsolutions
\printanswers



\begin{document}

%\Large

\def\mystrut{\protect\rule[-2.2ex]{0ex}{2.2ex}} 
\qformat{ \textbf{Task \#\thequestion}
  \ifthenelse{\equal{\thequestion}{\thequestiontitle}}
    {}
    {: \emph{\thequestiontitle}}
  \mystrut  \hfill}


\begin{questions}

\vspace*{\stretch{1}}

\question
  A 10-kg ball rolling down a hill accelerates from rest to 25~m/s in a time period of 2.7~seconds.  How much work was done if it rolled a total distance of 33.75~m?


  \begin{solution}
    \begin{align*}
      a &= \frac {25-0}{2.7} 
         = \SI{9.26}{\meter\per\second^2} \\
      F_{NET} &= (10)(9.26)
               = \SI{92.6}{\newton} \\
      W &= (92.6)(33.75)
         = \SI{3125.25}{\joule}
    \end{align*}


    Extension: what is the kinetic energy?  (should be same as work
  \end{solution}


\vs \hrule \vs

\question
  A certain cart has a mass of 450~kg and a kinetic energy of \SI{100000}{\joule}.  How far will it go in \SI{3.25}{\second}?

  \begin{solution}
      \begin{align*}
        100000 &= \frac{1}{2}(450)v^2 \\
        444.44 &= v^2 \\
        \SI{21.08}{\meter\per\second}  &= v \\\\
        %
        21.08 &= \frac{d}{3.25} \\
        \SI{68.51}{\meter} &- d
      \end{align*}
  \end{solution}

\vs
\ifprintanswers
  \hrule \vs
\else
  \pagebreak
  \vspace*{\stretch{1}}

  \newcommand{\eqsheet}{
    \section*{Equations}
    \begin{align*}
      v &= \frac{d}{t} &
      a &= \frac{\left(v_f-v_i\right)}{t} &
      F_{NET} &= ma \\\\\hline\\
      W &= Fd &
      KE &= \frac{1}{2}mv^2 &
      PE &= mgh
    \end{align*}
  }

  \eqsheet
  
  \vs \hrule \vs

  \eqsheet

  \vs 

\fi



\end{questions}




\end{document}