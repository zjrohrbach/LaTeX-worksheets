\documentclass[10pt]{exam}
\usepackage[icp]{template-for-exam}
\usepackage{multicol}
\usepackage{pgfplots}
\pgfplotsset{
  compat=1.18, 
  every axis/.append style={
    grid=major,
    xtick={0,1,...,25},
    ytick={-10,-9,...,15},
    grid style = dashed,
  },
  empgraph/.append style={
    xmin=0, xmax=9,
    ymin=0, ymax=6,
    width  = 5cm,
    height = 4cm,
    ylabel = {position},
    xlabel={time},
    xtick = \empty,
    ytick = \empty,
    axis y line = left,
    axis x line = bottom,
  },
  soln/.append style={
    red,
    domain=0:7
  }
}
\printanswers
\shadedsolutions

\title{P1-P3 Refresher}
\author{Rohrbach}
\date{\today}

\begin{document}
\maketitle

\begin{questions}

\uplevel{\section*{Unit P1: Motion}}

  \question
    Define the following terms: 
    \begin{parts}
        \part Velocity:
          %
          \begin{solution}[\stretch{1}]
            how fast an object is moving and its direction (measured in m/s)
          \end{solution}

        \part Acceleration:
          %
          \begin{solution}[\stretch{1}]
            the rate that velocity changes (measured in m/s/s or \SI{}{\meter\per\second^2})
          \end{solution}
    \end{parts}

  \question
    What is the acceleration of a car moving at a constant speed in a straight line?  How do you know?
    %
    \begin{solution}[\stretch{1}]
      zero.  The velocity is not changing.
    \end{solution}

  \question
    What are the three ways to accelerate?
    %
    \begin{solution}[\stretch{1}]
      speed up, slow down, change direction
    \end{solution}

    \question
    Draw the following distance-time graphs.

    \newcommand{\empgraph}[1]{
      \par
      \begin{tikzpicture}
        \begin{axis}[empgraph]
        \ifprintanswers
          \addplot[soln] {#1};
        \fi
        \end{axis}
      \end{tikzpicture}
    }

    \begin{multicols}{2}
      \begin{parts}
        \part Not moving \empgraph{3}
        \part Forward at a constant speed \empgraph{0.6*x}
        \part Backward at a constant speed \empgraph{5-0.6*x}
        \part Forward and speeding up \empgraph{(x/3)^2}
      \end{parts}
    \end{multicols}

\vspace{-2em}

\uplevel{\section*{Unit P2: Measurement}}

  \question

    %rowterminator
    \newcommand{\rt}{&\\\hline}
    %makecenter
    \newcommand{\ct}{\centering}

    \renewcommand{\arraystretch}{1.5}

    Fill in the blanks on this table:

    \begin{tabular}{|*{2}{m{.25\textwidth}|}m{.25\textwidth}c|}
      \hline
      \bf\ct Quantity & \bf\ct Tool & \bf\ct Units &\\\hline
                      &             & \ct mL       \rt
      \ct distance    &             &              \rt
                      &\ct thermometer &           \rt
                      &             & \ct seconds  \rt
                      & \ct balance &              \rt
    \end{tabular}

\pagebreak

  \question
    Round these numbers to two decimal places

    \begin{parts}
      \part 6.23842\vspace{2em}
      \part 0.31132\vspace{2em}
    \end{parts}

\uplevel{\section*{Unit P3: Forces \& Newton's Laws}}

\question
    What is {\bf inertia} and what law does it correspond to?

    \begin{solution}[\stretch{1}]
      Inertia is the tendency of object's to resist changes in motion.  It corresponds to Newton's First Law.
    \end{solution}
    
  \question
    Which of Newton's laws best explains each of these?  Explain your answer in at least one complete sentence.

    \begin{parts}
      \part 
        Jen goes shopping at the grocery store. She notices that as she adds items to the cart it gets harder to push.   

        \begin{solution}[\stretch{1}]
          Second Law.  As the mass of the cart increases, it accelerates less.
        \end{solution}

      \part 
        A rocket pushes fuel down so that the fuel can push the rocket up.

        \begin{solution}[\stretch{1}]
          Third law.  The action is the rocket pushing the fuel down; the reaction is the fuel pushing the rocket up.
        \end{solution}
    
      \part
        When you are in a car and you slam on your brakes, your body keeps moving forward.

        \begin{solution}[\stretch{1}]
          First law. Your body is in motion.  It tries to stay in motion even though the car stops
        \end{solution}

    \end{parts}

  \question
    You jump off the ground by pushing off of it.  The action force is the force of your feet pushing the ground down.  What is the reaction force?

        \begin{solution}[\stretch{1}]
          The force of the ground pushing your feet up.
        \end{solution}


  \question
    What is the difference between mass and weight?

    \begin{solution}[\stretch{1}]
      \begin{itemize}
        \item mass is a measure of an object's inertia
        \item weight is the force of gravity on the object
      \end{itemize}
    \end{solution}


	\question 
    If you go to a different planet, what happens to your mass and your weight?

    \begin{solution}[\stretch{1}]
      Your mass stays the same, but your weight changes.
    \end{solution}


    



\end{questions}
\end{document}