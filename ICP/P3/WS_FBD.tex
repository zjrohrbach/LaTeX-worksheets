\documentclass[10pt]{exam}
\usepackage[icp]{template-for-exam}
\usepackage{my-tikz-clipart}
\usetikzlibrary{patterns.meta}

\title{Free-Body Diagram Practice}
\author{Rohrbach}
\date{\today}




\begin{document}
\maketitle

\begin{questions}
  
  \question
    A cat sits at rest on the top of a table.
  
    \begin{parts}
      
      \part
        Will the forces on the cat be balanced or unbalanced?  How do you know?
        \vs

      \part
        Draw the free body diagram.  
        
        Make sure to include the net force.
        
        \begin{center}
          \begin{tikzpicture}
            \draw[pattern=north east lines]
              (-2,.1) rectangle (2,-.1);
            \cat[contour=blue,scale=0.7]
          \end{tikzpicture}
        \end{center}
        \vspace{1em}


      \part
        If the cat's weight (that is, force of gravity) is 90 Newtons, what is the normal force acting on the cat?
        \vs

    \end{parts}

  \hrule

  \question
    A car is moving east along the interstate {\bf at a 
    constant speed}.

    \begin{parts}
  
      \part
        Will the forces on the car be balanced or unbalanced? How do you know?
        \vs
    
      \part
        Draw the free body diagram.  
        
        Make sure to include the net force.
        
        \cardrawing
        \vspace{2em}
      
      \part
        The car has an applied force of 700 Newtons.  Find the magnitude of the force of {\bf friction}.
        \vs

    \end{parts}


  \hrule

  \question
    The same car is now {\bf accelerating forward}.

    \begin{parts}
      
      \part
        Will the forces on the car be balanced or unbalanced? How do you know?
        \vs
    
      \part
        Draw the free body diagram.  
        
        Make sure to include the net force.
        
        \cardrawing
        \vspace{2em}

    
      \part
        The frictional force is the same as in the last problem, but this time, the car has an applied force of 900 Newtons.  Calculate the {\bf net force}.
        \vs

    \end{parts}


  \pagebreak

  \question
    You are lifting a bucket with a rope.  The force on 
    the rope is 45 Newtons and the bucket has a weight
    (that is, force of gravity) of 23 Newtons.
    
    \begin{parts}

      \part
        Draw the free body diagram.  
        
        Make sure to include the net force.

        \begin{center}
          \begin{tikzpicture}
            \path (0,-1.8) pic {bucket};
            \filldraw[pattern=north east lines] 
            (-.1,0) -- (.1,0) -- (.1,1) -- (-.1,1) -- cycle;
          \end{tikzpicture}
        \end{center}
        
      \part
        Calculate the {\bf net force} on the bucket.
        \vs

    \end{parts}

  \hrule

  \question 
    The net force acting on a bicycle is 14 N.  The friction experienced by the bicylce is 52 N.

    \begin{parts}
    
      \part 
        Draw a free body diagram.  
        

        \begin{center}
          \begin{tikzpicture}
            \draw[pattern=north east lines] 
              (-6,0) rectangle (7,-.2);
            \path (-1,.75) pic[scale=0.7] {bike};
            \draw[ultra thin, draw=gray!50] 
              (-2.5,1.5) -- + (-1.5,0) 
              ++ (-.5,-.3) -- + (-1.25,0) 
              ++ (-.1,-.3) -- + (-.8,0);
          \end{tikzpicture}
        \end{center}
      
      \part
        Calculate the {\bf applied force} on the bicycle.
        \vs

    \end{parts}

  \hrule
  
  \question 
    A large load has a weight (that is, force of gravity) of \SI{12000}{\newton}.  It is being lifted by a crane.

    \begin{parts}
    
      \part 
        Draw a free body diagram.  
        

        \begin{center}
          \begin{tikzpicture}
            \draw[
              orange, 
              thick, 
              pattern={Hatch[angle=45,distance={6pt},line width=0pt,]},
              pattern color=orange
              ] 
                (0,0) -- 
                ++(3,4) -- 
                ++(.1,0) coordinate (top) --
                ++(.1,0) --
                (1,0) --
                cycle;
            \filldraw[pattern=north east lines] 
              (top) -- 
              ++(-.05,0) --
              ++(0,-2) coordinate (a) -- 
              ++(.05,0) coordinate (b) -- 
              ++ (0,1.95) --
              cycle;
            \draw (a) -- ++(-.5,-.5) coordinate (c);
            \draw (b) -- ++(.5,-.5) coordinate (d);
            \draw[
                red, 
                thick,
                pattern=bricks, 
                pattern color=red
              ] (c) -- ++(0,-1) -| (d) -- cycle;
          \end{tikzpicture}
        \end{center}
      
      \part
        The net force is 1500 N upward.  Calculate the {\bf tension} being provided by the crane.
        \vs

    \end{parts}
    
    
  
  
  
  
  
  
  
  

\end{questions}

\end{document}