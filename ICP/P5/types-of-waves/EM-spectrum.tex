\documentclass[landscape]{article}
\usepackage[margin=0.5in]{geometry}
\usepackage{pgfplots}
\pgfplotsset{compat=1.18}
\pgfdeclareverticalshading{rainbow}{100bp}
 {
  color(0bp)=(red); 
  color(25bp)=(red); 
  color(35bp)=(yellow);
  color(45bp)=(green); 
  color(55bp)=(cyan); 
  color(65bp)=(blue);
  color(75bp)=(violet); 
  color(100bp)=(violet)
}
\usetikzlibrary{arrows.meta}

\begin{document}
\pagestyle{empty}

\newcommand{\wave}{
  \noindent
  \begin{tikzpicture}[x=1cm]
    \def\tickheight{.2}
    \node[anchor=west] at (-6,4) 
      {\bf \Large The Electromagnetic Spectrum};
    \draw[<->,thick] (-6,0) -- (6,0);
    \foreach \x in {-4.5,-2.25,-.1,.1,2.25,4.5} {
      \draw (-\x,0) +(0,\tickheight) -- +(0,-\tickheight);
    }
    \shade[shading=rainbow,shading angle=-90] 
      (-.1,\tickheight) 
      -- (.1,\tickheight) 
      -- (.1,-\tickheight) coordinate (right vis)
      -- (-.1,-\tickheight) coordinate (left vis)
      -- cycle;
    \shade[shading=rainbow,shading angle=-90] 
        (-4,-1) coordinate (left zoom) 
      -- (4,-1) coordinate (right zoom)
      -- (4,-2) coordinate (bright zoom)
      -- (-4,-2) coordinate (bleft zoom)
      -- cycle;

    \draw (left vis) -- (left zoom) -- (bleft zoom)
      -- (bright zoom) -- (right zoom) -- (right vis);

    \begin{scope}[shift={(-6,2)}]
      \begin{axis}[
          red,
          width=12cm,
          height=5mm,
          scale only axis,
          ultra thick,
          xmin=-600,
          xmax=-1,
          hide axis
        ] 
        \addplot[domain=-600:-1,samples=900]{sin(150*sqrt(-x))};
      \end{axis}
    \end{scope}

  \end{tikzpicture}
}

\newcommand{\myline}{
  \wave\hspace{\stretch{1}}\wave
}



\myline\vspace{0.5in} \vspace{\stretch{1}}

\vspace{0.5in}\myline \vspace{\stretch{1}}



\end{document}