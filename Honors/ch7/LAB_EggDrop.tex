\documentclass[10pt]{exam}
\usepackage[hon]{template-for-exam}

\title{Egg Drop Lab}
\author{Rohrbach}
\date{\today}

\begin{document}
\maketitle

\section*{Objective}

The goal is to build a device that will protect and egg dropped off the balcony in the Commons and prevent it from breaking.

\section*{Rules}

\begin{enumerate}
  \item This is an \emph{individual} project
  \item The egg and device must weigh less than 200 grams (the average egg weighs around 50 grams)
  \item You must use my egg when you get here. (\emph{i.e.} no soaking in vinegar, no hard-boiling, etc.)
  \item You may not use bubble wrap
  \item Your entire device must not be taller than 75 cm.
  \item Once a device has hit the ground, it will be your responsibility to retrieve your egg and prove that it did not break.
  \item The time does not stop until the device come to a rest!
  \item Drop day will be on Fri, Jan 24.  If your egg does not survive, you will have a second chance on Mon, Jan 27.
\end{enumerate}

\section*{Winners}

Bonus points will be awarded to the top three eggs that survive and have the highest score according to this equation:
%
\begin{align*}
  S &= \frac{\SI{10000}{}}{m\cdot t\cdot r} \,\,\,,
\end{align*}
%
where $m$ is the mass, $t$ is the time of fall (not to be confused with time of collision $\Delta t$), and $r$ is the distance that the egg lands relative to the target.


\section*{Lab Writeup}

On Schoology, you will turn in a Lab Writeup on a Word document that addresses the following:


\begin{enumerate}
  \item {\bf Purpose}
  \item {\bf Your design.} Explain your design and how you expected it to work.  Make sure to explain how your device intends to address momentum and/or impulse.
  \item {\bf Results.} After you drop the egg, explain what happened. Were you successful? What was your score?
  \item {\bf Conclusion.}  Explain how the results differed from what you expected and what you could do differently to improve your design.  Make sure to draw a connection between your suggested improvements and the concepts of impulse or momentum.
\end{enumerate}


\section*{Grading Rubric}

\paragraph{Preparedness}  You are prepared on Drop Day
 
$\circ$ 2/2 You are ready to go on drop day \hfill
$\circ$ 0/2 You were late \hfill{}

\paragraph{Success}  Your egg did not break.
 
$\circ$ 4/4 Your egg survived \hfill
$\circ$ 0/4 Your egg broke \hfill{}

\paragraph{Application to Momentum and Impulse} Your design discussion and your conclusion discussion accurately and completely draws the connection between the lab and impulse/momentum concepts.

$\circ$ 10/10 Meets Standard \hfill

$\circ$ 7/10 You address impulse/momentum, but your discussion is not quite right \hfill

$\circ$ 5/10 Impulse/Momentum is superficially addressed \hfill

$\circ$ 0/10 You do not address momentum or impulse \hfill


\paragraph{Completeness}  Your design and conclusion sections show that you have put some thought into this

$\circ$ 4/4 Meets Standard \hfill

$\circ$ 3/4 Needs work \hfill

$\circ$ 0/4 These sections are missing or extremely sparse on details \hfill



\paragraph{Bonus}  You were in the top three for the class.

$\circ$ +3 First Place \hfill
$\circ$ +2 Second Place \hfill
$\circ$ +1 Third Place \hfill{}



\paragraph{Total Score} \fillin[][5em] / 20


\end{document}