\documentclass[10pt]{exam}
\usepackage[hon]{template-for-exam}
\usepackage{tikz,tikzpingus,graphicx}
\usetikzlibrary{shadings,decorations.pathmorphing,arrows.meta,patterns}


\title{The Elevator Effect}
\author{Rohrbach}
\date{\today}

\begin{document}
\maketitle

\noindent
When do you are riding in an elevator, there are some times that you feel heavier and some times that you feel lighter.

\vspace{1em}

\begin{parts}

  \part Draw a Free-Body Diagram for each of these situations:

  \def\headsize{0.4}
  \def\elevatorwidth{2.5}

  \tikzstyle{person}=[ultra thick, orange!50, scale=0.8]

  \tikzset{
    elevator/.pic = {
      \draw[fill=gray!30] 
        (-.2,-.05) rectangle (\elevatorwidth+.2,4.05);
      \draw[fill=white] 
        (0,0) rectangle (\elevatorwidth,4);

      %guide-line (for development)
      %\draw (\elevatorwidth/2,0) -- ++(0,4);


      \path (\elevatorwidth/2,4.4) coordinate (pulley);
      \draw (pulley) -- ++(0,-0.4);
      \filldraw[fill=gray] (pulley) circle (0.3);
      \filldraw[fill=gray!50] (pulley) circle (0.2);
      \draw[thick] (pulley) 
        ++(-0.3,1) --
        ++(0,-1)
        arc[start angle=-180, end angle=0, radius=0.3] --
        ++(0,1);
    }
  }


  \begin{center}
    \begin{tikzpicture}

      \begin{scope}
        \path (0,0) pic{elevator} coordinate (start);

        \begin{scope}[person]
          \draw (start) 
          ++(.95,0) coordinate (left foot) --
          ++(60:1.2) coordinate (waist) --
          ++(-60:1.2) coordinate (right foot);

          \draw (waist)
            -- ++(90:1.5) coordinate (neck);
          \draw (neck) 
            ++(90:\headsize) circle (\headsize);
          \draw (neck)
            -- ++(-70:0.7) coordinate (right elbow)
            -- ++(-100:0.5) coordinate (right hand);
          \draw (neck)
            -- ++(-110:0.7) coordinate (left elbow)
            -- ++(-80:0.5) coordinate (left hand);
        \end{scope}

        \draw (start) ++(\elevatorwidth/2,-0.5)
          node {\footnotesize stationary elevator};

      \end{scope}

      \begin{scope}
        \path (4.2,0) pic{elevator} coordinate (start);


        \begin{scope}[person]
          \draw (start) 
          ++(1.2,0) coordinate (left foot) --
          ++(110:0.6) --
          ++(20:0.6) coordinate (waist) --
          ++(-20:0.6) --
          ++(250:0.6) coordinate (right foot);


          \draw (waist)
            -- ++(90:1.5) coordinate (neck);
          \draw (neck) 
            ++(90:\headsize) circle (\headsize);
          \draw (neck)
            -- ++(-10:0.7) coordinate (right elbow)
            -- ++(10:0.5) coordinate (right hand);
          \draw (neck)
            -- ++(190:0.7) coordinate (left elbow)
            -- ++(170:0.5) coordinate (left hand);
        \end{scope}

        \draw (start) ++(\elevatorwidth/2,-0.5)
          node {\footnotesize accelerating upward};

      \end{scope}

      \begin{scope}
        \path (8.4,0) pic{elevator} coordinate (start);

        \begin{scope}[person]
          \draw (start) 
          ++(1.2,0) coordinate (left foot) --
          ++(75:1.2) coordinate (waist) --
          ++(-20:1.2) coordinate (right foot);

          \draw (waist)
            -- ++(100:1.5) coordinate (neck);
          \draw (neck) 
            ++(100:\headsize) circle (\headsize);
          \draw (neck)
            -- ++(10:0.7) coordinate (right elbow)
            -- ++(80:0.5) coordinate (right hand);
          \draw (neck)
            -- ++(-160:0.7) coordinate (left elbow)
            -- ++(110:0.5) coordinate (left hand);
        \end{scope}

        \draw (start) ++(\elevatorwidth/2,-0.5)
          node {\footnotesize accelerating downward};

      \end{scope}

    \end{tikzpicture}
  \end{center}

  \vs[3]

  \part Write an expression for the normal force in each situation above.
  \vs[2]

  \part Which force corresponds to ``how heavy'' you feel?
  \vs

  \part How could you use an elevator to simulate weightlessness?
  \vs
\end{parts}


\end{document}