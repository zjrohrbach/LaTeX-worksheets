\documentclass[10pt]{exam}
\usepackage[hon]{template-for-exam}
\usepackage{enumitem}
\usepackage{tikz}
\usepackage{multicol}
\usetikzlibrary{shadings,decorations.pathmorphing,arrows.meta}



\def\mytitle{Chapter 4 (Dynamics)}
\author{Rohrbach}
\date{\today}

\def\mymaketitle{
  \begin{flushleft}
    {\LARGE \textbf \mytitle \par}
  \end{flushleft}
}



\begin{document}


\mymaketitle



\newcommand{\stampbox}[1]{

  \hfill
  \begin{tikzpicture}[every text node part/.style={align=center}]
     \node[gray!50,draw,rounded corners] at (0,0) 
      {\sc Stamp \\ \sc Here \\ \small #1 \sc Points};
  \end{tikzpicture}
  \vspace{1em}
  
  \hrule

}

\section*{Homework Check A (collected October 11)}


%%%%%%%%%%%


\paragraph{Intro to Forces} p. 101 \#7, 9
\dotfill Complete by Mon, Oct 7

\stampbox{2}


%%%%%%%%%%%

\paragraph{Single-Body Dynamics} p. 101 \#10, 12, 17, 18
\dotfill Complete by Mon, Oct 7


\stampbox{5}

%%%%%%%%%%%%

\paragraph{Multi-Body Dynamics} pp. 102-103, 106 \#20, 25, 33ab, 79, 81
\dotfill Complete by Fri, Oct 11

\hfill \textbf{\emph{Homework Quiz}}


\stampbox{5}
  

%%%%%%%%%%%

\paragraph{Free-Body Diagrams} pp. 102, 104 \#21, 22, 52
\dotfill Complete by Fri, Oct 11

{\sc Draw the free-body diagrams only; nothing to calculate}

\stampbox{3}
  

%%%%%%%%%%%%%

\subsection*{Answers}

\begin{multicols}{3}

  \begin{itemize}[noitemsep]
    \item[7. ] 3134 N
    \item[9. ] 779.5 N
    \item[10.] 12,600 N
    \item[12.] 1.84 m/s$^2$
    \item[17.] (a) 7.35 m/s$^2$; (b) 1293.6 N
    \item[18.] 0.44 m/s$^2$
    \item[20.] (a) 47 N; (b) 17 N; (c) 0 N
    \item[25.] (a) 31.36 N, 62.72 N; \\
               (b) 35.36 N, 70.72 N
    \item[33.] (a) 2.72 m/s$^2$ (b) 0.96 s
    \item[79.] (a) 87,556 N; \\
               (b) 11,448 N; \\
               (c) 11,448 N, down
    \item[81.] (a) either 45 N or 4.6 kg; \\
               (b) 37.4 N or 3.8 kg; \\
               (c) No. the minimum force needed to lift a 15-lb fish would be 15 lbs.
    
    
  \end{itemize}
  
\end{multicols}

\noindent
{\footnotesize Homework will be accepted for full credit until the test.
Homework turned in after the test will be accepted for half credit
until the Unit 3 Test.
\emph{Please remember that you will not be eligible to complete 
test corrections if you do not turn in your homework.}}

\vspace{1em}
\hrule 


%%%%%%%%%%%%%
%%%%%%%%%%%%%

\pagebreak

\mymaketitle

\section*{Homework Check B (collected on Test Day)}

\paragraph{Friction} pp. 103-104 \#36, 37, 42, 44
\dotfill Complete by Fri, Nov 1


\stampbox{5}

%%%%%%%%%%%%%%

\paragraph{Forces at an Angle} pp. 102-104 \#50abc, 56%#, 57
\dotfill Complete by Mon, Nov 4
   
\stampbox{5}

%%%%%%%%%%%%%%

\paragraph{Conceptual Questions} pp. 98-99 \#1, 3, 6, 7, 10, 11, 12, 13
\dotfill Complete by Mon, Nov 4
   
{\sc These questions should have at least one full sentence 
      of explanation}

\stampbox{5}

%%%%%%%%%%%%%%

\paragraph{Misconceptual Questions} pp. 99-100 \#1, 2, 3, 5, 6, 7, 9
\dotfill Complete by Mon, Nov 4
   
{\sc You do not need to get this one stamped,
but these are good review for your test!}

\vspace{1em}
\hrule

%%%%%%%%%%%%%%

\paragraph{Bonus Problems!} \#26, 34, 35, 49, 60
\dotfill Turn in separately on test day!

\vspace{1em}
\hrule


%%%%%%%%%%%%%%

\paragraph{Test will be on Tue, Nov 5 (Pd 1) / Wed, Nov 6 (Pd 7).} \hfill


\subsection*{Problem Answers}

\begin{multicols}{2}

  \begin{itemize}[noitemsep]
    \item[36.] (a) 64.7 N; (b) 0 N
    \item[37.] (a) 0.60; (b) 0.53
    \item[42.] 4.17 m
    \item[44.] 33.6 m/s
    \item[50.] (b) 62.2 N; \\
               (c) 199.4 N; \\
               (d) $F_{Ax}=70.6$~N, $F_A=99.8$~N
    \item[56.] $F_f=104$ N; $\mu=0.48$
    %\item[57.] (a) 3.7 m/s/s; (b) 9.4 m/s

    
  \end{itemize}
  
\end{multicols}

\subsection*{Misconceptual Answers}

\begin{multicols}{7}

  \begin{itemize}[noitemsep]
    \item[1.] a
    \item[2.] abcd
    \item[3.] d
    \item[5.] c
    \item[6.] c
    \item[7.] c
    \item[9.] c
  \end{itemize}
  
\end{multicols}

\hrule
\vspace{0.2em}
\hrule

%%%%%%%%%%%%%%

\section*{Extra Practice}

These problems are not required and are not for bonus.  Work and answers are available on Schoology.
  
Single-Body \dotfill p. 101 \#18; p. 103 \#37b

Multi-Body \dotfill p. 103 \#33a

\end{document}