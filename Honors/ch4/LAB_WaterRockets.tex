\documentclass[10pt]{exam}
\usepackage[hon]{template-for-exam}
\usepackage{enumitem,multicol}
\setlist[itemize]{itemsep=-1ex,partopsep=1ex,parsep=1ex}

\title{Water Rockets}
\author{Rohrbach}
\date{\today}

\begin{document}
\maketitle

\paragraph{Purpose:}  
  The goal of this lab is to build a water rocket that will remain in the air the longest.

\paragraph{Rules:}
  You may use any household materials to turn your ordinary 2-liter bottle into a rocket.  The minimum requirements are that your rocket must have:
    \begin{itemize}
      \item one unaltered 2-liter bottle as a ``fuel tank''
      \item a cone
      \item fins
    \end{itemize}
  
  You should not puncture any holes in your 2-liter bottle.  It should not be altered in any way other than adding materials to the outside. The opening of the bottle should be \emph{pointing downward} and uncovered. Try to avoid adding things around the neck of the bottle that would prevent the launchpad from attaching properly.

  You may add a parachute, wings, bubble wrap, or any household material.  Be creative, but realistic as well.  You must provide all of your own supplies and will not be given any in-class time to do the building of your rocket. 

  \begin{center}
    \begin{tabular}{|c|}
      \hline \\
      \bf 
      Date to bring your completed rocket to class:
      \fillin[][10em]\\
      \hline
    \end{tabular}
  \end{center}

\paragraph{Launch Day:}
  On launch day, you will need to fill your bottle between 25\% and 75\% with water.  The actual amount is up to you.  Then, we will attach your rocket to a launch pad and air pump. You will then use an air pump to raise the internal pressure to 80 PSI and then release the rocket when instructed.  The winner will be the rocket that stays in the air the longest, as measured from time of launch until time it strikes the ground. 





\paragraph{Planning:}
  Other than the sketch, your answers to each question should be written in complete sentences.  This entire ``Planning'' section must be complete and your rocket must be built by launch day.

\begin{questions}
  \question
    Draw a sketch of your proposed design.  Make sure to label all of the parts on your drawing (be specific like what materials you're using and where things will go).
    \vs[2]

  \question 
    Explain why you decided to do the design you chose.  In other words, what parts of your design do you think are particularly good, and why do you think they will work well? 
    \vs

  \pagebreak
  
  \question
    Explain how Newton's First Law applies to your rocket.  How might understanding this law help you to build a better rocket?
    \vs
  
  \question
    Explain how Newton's Second Law applies to your rocket.  How might understanding this law help you to build a better rocket?
    \vs

  \question
    Explain how Newton's Third Law applies to your rocket.  How might understanding this law help you to build a better rocket?
    \vs
  
  \begin{EnvUplevel}
    \paragraph{Launch Day:}
      Time in the air: \fillin[][4em] seconds.
  \end{EnvUplevel}

  \question
    Describe what actually happened during the launch of your rocket.  Make sure to mention anything that did not go as expected.
    \vs[2]
  
  \question
    Think about Newton's laws and use them to explain what worked and what didn't work on your rocket.
    \vs


\end{questions}

\hrule
\paragraph{Grading Rubric:} \hfill

\begin{multicols}{3}

  \small
  Prepared for launch day?
  
  \begin{tabular}{crl}
    $\bigcirc$ & \bf 10: & yes \\
    $\bigcirc$ & \bf 5: & launched late \\
    $\bigcirc$ & \bf 0: & never launched \\
  \end{tabular}

  Worksheet complete?
  
  \begin{tabular}{crl}
    $\bigcirc$ & \bf 10:& thorough \& complete \\
    $\bigcirc$ & \bf 7: & needs more detail \\
    $\bigcirc$ & \bf 5: & glaring errors/incomplete  \\
  \end{tabular}

  Bonus points
  
  \begin{tabular}{crl}
    $\bigcirc$ & \bf +3: & first place \\
    $\bigcirc$ & \bf +2: & second place \\
    $\bigcirc$ & \bf +1: & third place \\
  \end{tabular}

\end{multicols}
%
\begin{flushright}
  \small 
  Total Score:   \fillin[][5em]/20
\end{flushright}

\end{document}