\documentclass[10pt]{exam}
\usepackage[hon]{template-for-exam}
\usepackage{tikz}
\usetikzlibrary{shadings,decorations.pathmorphing,arrows.meta,patterns}

\title{Friction Practice Problems}
\author{Rohrbach}
\date{\today}

\begin{document}
\maketitle

\begin{questions}

\question 
  A 75-kg bookcase is pushed across a carpeted floor.  The static coefficient of friction is 0.8 and the kinetic coefficient of friction is 0.6.

  \begin{parts}
    \part 
      How much force is required to start the bookcase moving?
      \vs
    \part 
      If that same force is continually applied to the bookcase after it starts moving, what will be its acceleration?
      \vs
  \end{parts}

\pagebreak

\question
    Consider the modified Atwood Machine shown with $m_1=4$~kg and $m_2=3$~kg.  If the coefficient of kinetic friction between the top box and the surface of the table is 0.2, what will be the acceleration of the system?

    \tikzstyle{box}=[rounded corners,draw,minimum height=1cm]


    \begin{tikzpicture}
      \node[box,minimum width=1.5cm] (one) at (0,0) {$m_1$};
      \node[box] (two) at (3.75,-3) {$m_2$};
      \draw[very thick] (one.east) -- (3.5,0) arc (90:0:0.25) -- (two);
      \fill[pattern=north east lines] (-3,-0.5) rectangle
        ++(6,-0.25);
      \draw (-3,-0.5) -- 
        ++(6,0) coordinate (edge) -- 
        ++(0,-1);
      \draw (edge) -- (3.5,-0.25) circle (0.25);
    \end{tikzpicture}



\end{questions}

\end{document}