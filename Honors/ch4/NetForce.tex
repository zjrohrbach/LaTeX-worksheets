\documentclass[10pt]{exam}
\usepackage[hon]{template-for-exam}
\usepackage{multicol}
\usepackage{tikz}
\usetikzlibrary{arrows.meta}
\tikzset{f/.append style={-{Stealth[length=3mm,width=2mm]}}}
\usepackage{graphicx}
\graphicspath{{./FBDs}}

\title{Net Force and Free-Body Diagrams}
\author{Rohrbach}
\date{\today}

\newcommand{\hforce}[2]
{
  \draw[f] (a) -- ++(#1, 0) 
        node[anchor=south] {#2};
}
\newcommand{\vforce}[2]
{
  \draw[f] (a) -- ++(0, #1) 
        node[anchor=west] {#2};
}

\begin{document}
\maketitle

\begin{questions}

  \question 
  In each of the free-body diagrams below, calculate the 
  {\bf magnitude} and {\bf direction} of the net force 
  and draw it.


  \begin{center}  
    \begin{tikzpicture}[scale=0.5]
      \newcommand{\createFBD}[1]
        {
          \filldraw (a) circle [radius=0.15];
          \path (a) node {} 
            + (-3,2.5) node {#1} ;
        }

      \begin{scope}
        \node (a) at (0,0) {};
        \createFBD{(a)}
        \vforce{ 2}{\SI{18}{\newton}}
        \vforce{-2}{\SI{18}{\newton}}

      \end{scope}

      \begin{scope}
        \node (a) at (8,0) {};
        \createFBD{(b)}
        \vforce{  2  }{\SI{40}{\newton}}
        \vforce{-1.25}{\SI{24}{\newton}}
      \end{scope}

      \begin{scope}
        \node (a) at (16,0) {};
        \createFBD{(c)}
        \vforce{ 2}{\SI{5}{\newton}}
        \vforce{-2}{\SI{5}{\newton}}
        \hforce{ 3}{\SI{7}{\newton}}
        \hforce{-3}{\SI{7}{\newton}}
      \end{scope}

      \begin{scope}
        \node (a) at (24,0) {};
        \createFBD{(d)}
        \vforce{  2 }{\SI{17}{\newton}}
        \vforce{ -2 }{\SI{17}{\newton}}
        \hforce{  3 }{\SI{32}{\newton}}
        \hforce{-1.5}{\SI{14}{\newton}}
      \end{scope}
    \end{tikzpicture}
  \end{center}


      
\question 
  In each of the free-body diagrams below, the net force 
  is given, but one or more of the applied forces is 
  missing.  Find the missing forces.

  \begin{EnvUplevel}
    \begin{tikzpicture}[scale=0.77]
      \def\sep{5.2}

      \newcommand{\createFBD}[2]
        {
          \filldraw (a) circle [radius=0.15];
          \path (a) node {} 
            + (0,-3) node {\footnotesize
              $\Sigma\vec{F}=$ #2
            }
            + (-1.5,2.5) node {#1} ;
        }
      \newcommand{\forceblank}
        {\,\fillin[][3em]\SI{}{\newton}}


      \begin{scope}
        \node (a) at (0,0) {};
        \createFBD{(a)}{\SI{0}{\newton}}
        \vforce{  2 }{\forceblank}
        \vforce{ -2 }{\SI{20}{\newton}}
        \hforce{ 1.5}{\SI{15}{\newton}}
        \hforce{-1.5}{\forceblank}
      \end{scope}

      \begin{scope}
        \node (a) at (\sep,0) {};
        \vforce{  2 }{\forceblank}
        \vforce{ -1 }{\SI{200}{\newton}}
        \createFBD{(b)}{\SI{150}{\newton}, up}
      \end{scope}

      \begin{scope}
        \node (a) at (2*\sep,0) {};
        \createFBD{(c)}{\SI{45}{\newton}, left}
        \vforce{  1.5 }{\SI{60}{\newton}}
        \vforce{ -1.5 }{\forceblank}
        \hforce{1.25}{\forceblank}
        \hforce{ -2 }{\SI{80}{\newton}}
      \end{scope}

      \begin{scope}
        \node (a) at (3*\sep,0) {};
        \createFBD{(d)}{\SI{23}{\newton}, right}
        \vforce{  2 }{\forceblank}
        \vforce{ -2 }{\forceblank}
        \hforce{ 1.7}{\forceblank}
        \hforce{ -1 }{\SI{36}{\newton}}
      \end{scope}

    \end{tikzpicture}
  \end{EnvUplevel}

\question
  Fill in the blanks in each of the situations depicted 
  below.  Draw the net force.

  \begin{center}  
    \begin{tikzpicture}[scale=0.6]
      \newcommand{\createFBD}[4]
        {
          \filldraw (a) circle [radius=0.15];
          \path (a) node {} 
            + (-3,2.5) node {#1}
            ++ (-1,-3.5) node[anchor=east] {$m=$}
            ++ (0,  0) node[anchor=west] {#2}
            ++ (0,-.8) node[anchor=east] {$a=$}
            ++ (0,  0) node[anchor=west] {#3}
            ++ (0,-.8) node[anchor=east] {$\Sigma\vec{F}=$ }
            ++ (0,  0) node[anchor=west] {#4}
            ;
        }
      \newcommand{\makeblank}{\fillin[][3em]{} }

      \newcommand{\forceblank}
        {\,\makeblank N}

      \draw[dashed] (0,6) -- (0,-9);
      \draw[dashed] (12,-1.5) -- (-12,-1.5);


      \begin{scope}
        \node (a) at (-7,4) {};
        \createFBD
          {(a)}{3 kg}{\makeblank m/s$^2$, \makeblank}{23 N, right}
        \vforce{  2 }{\forceblank}
        \vforce{ -2 }{29.4 N}
        \hforce{ 3.5}{\forceblank}
        \hforce{ -1}{5 N}

      \end{scope}

      \begin{scope}
        \node (a) at (7,4) {};
        \createFBD
          {(b)}{2 kg}{8 m/s$^2$, left}{\makeblank N, \makeblank}
        \vforce{  2 }{19.6 N}
        \vforce{ -2 }{\forceblank}
        \hforce{-1.8}{\forceblank}
      \end{scope}

      \begin{scope}
        \node (a) at (-7,-5) {};
        \createFBD
          {(c)}{5 kg}{12 m/s$^2$, left}{\makeblank N, \makeblank}
        \vforce{  1.5 }{49 N}
        \vforce{ -1.5 }{\forceblank}
        \hforce{  1   }{7 N}
        \hforce{ -3   }{\forceblank}
      \end{scope}

      \begin{scope}
        \node (a) at (7,-5) {};
        \createFBD
          {(d)}{3 kg}{18 m/s$^2$, right}{\makeblank N, \makeblank}
        \vforce{  2.5 }{\forceblank}
        \vforce{ -2.5 }{\forceblank}
        \hforce{  4   }{\forceblank}
        \hforce{  -2   }{17 N}
      \end{scope}
    \end{tikzpicture}
  \end{center}

\pagebreak

\question
  For each of the sketches below, identify all the forces applied on all objects and draw a free body diagram.  Then come up with an expression for the net force.

  \newcommand{\picframe}[1]{
    \begin{center}
      \begin{tikzpicture}
        \clip (-3,-3) rectangle (3,3);
        \node at (0,0) {\includegraphics{#1}};
      \end{tikzpicture}
    \end{center}
  }

  \begin{multicols}{2}
    \begin{parts}
      \part 
        Lamp hanging from a chain

        \picframe{light.png}

      \part 
        A car moving at a constant speed.

        \picframe{car.png}

      
      \part 
        A car accelerating

        \picframe{car.png}

      \columnbreak

      \part 
        A box being pushed forward on the ground (constant speed)

        \picframe{box.png}

      \part 
        A skydiver before opening her parachute

        \picframe{free-fall-diver.png}


      \part 
        Object sliding down an inclined plane.

        \picframe{incline.png}

      
    \end{parts}
    
  \end{multicols}

  
\end{questions}
\end{document}