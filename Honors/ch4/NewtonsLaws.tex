\documentclass[10pt]{exam}
\usepackage[hon]{template-for-exam}

\title{Newton's Laws of Motion}
\author{Rohrbach}
\date{\today}

\begin{document}
\maketitle

\section*{Definitions}

\begin{itemize}
  \item kinematics
  \item dynamics
  \item directly proportional
  \item inversely proportional
  \item force
  \item free-body diagram
  \item inertia
  \item mass
  \item weight
\end{itemize}


\section*{Newton's Laws}

\subsection*{Newton's First Law (NFL)}

Every object \fillin[continues][7em] in its state of \fillin[continues][5em] or \fillin[uniform velocity][10em] in a straight line as long as no net force acts on it.

\vspace{1em}


\subsection*{Newton's Second Law (NSL)}

The acceleration of an object is
%
\begin{itemize}
  \item directly proportional to \fillin[the net force][10em] acting on it.
  \item inversely proportional to its \fillin[mass][5em]
  \item pointing in the \fillin[same direction][12em] as the net force
\end{itemize}

\vspace{5em}


\subsection*{Newton's Third Law (NTL)}

Whenever one object exerts a force on a second object, the second objec exerts an \fillin[equal][7em] force on the first object in the \fillin[opposite][7em] direction.

\vspace{5em}

\noindent
Short form: 




\end{document}