\documentclass[10pt]{exam}
\usepackage[hon]{template-for-exam}
\usepackage{enumitem}
\setlist[enumerate,itemize]{topsep=0pt,itemsep=-1ex,partopsep=1ex,parsep=1ex}
\usepackage{multicol}

\title{Lab -- Air Rockets!}
\author{Rohrbach}
\date{\today}

\begin{document}
\maketitle

\section*{Problem} You will be given a target.  The goal is to hit the target with an air rocket in three launches. The air rockets have adjustable pressure and an adjustable launch angle.


\section*{Pre-Lab} You only have three launches so think about what you are going to be doing with each trial.  What calculations do you need to make each time?

\vs


\section*{Data and Calculations} Make sure to record the angle you use, the pressure you use, and the range.  Also include any calculations.  For Trials \#2 \& 3, make sure to jot down a few notes about what changes you've made since the previous trial and why.

\vspace{2em}

Trial \#1:  
\vs

Trial \#2:
\vs

Trial \#3:  
\vs

\pagebreak


\section*{Lab Report} You will not be turning in anything hand written.  Anything important that you wrote down needs to end up in your lab report that you will submit electronically on Schoology.

\subsection*{Section I: Statement of the problem}
In your own words, what are you trying to accomplish in this lab?  This can be brief: probably 1-3 sentences.  Be as obvious and specific as you can.  What are you specifically trying to do?  Don't say something vague, like ``The purpose of this lab was to learn about projectile motion.''

\subsection*{Section II: Procedure and Data}
This should be a narrative that explains what did with each launch and why.  For each launch, list the pressure you used, the angle, and the range of the projectile (you may present this either as a narrative or in a data table.)  Also, make sure to explain what changes you made after each trial and why.  This explanation should include the calculations you made.

\subsection*{Section III: Conclusion}
Your conclusion should do the following things in narrative form:
\begin{itemize}
  \item[a.] Explain how successful the experiment was.
  \item[b.] Explain where errors came from. Be specific—you should never use vague allusions to “human error.”  What could be done in the future to improve your results?
  \item[c.] Explain what you learned in the lab
\end{itemize}


\section*{Grading Rubric}


\paragraph{Purpose} You have a purpose statement and it accurately matches the lab.	
\begin{itemize}
  \item[\bf 5:] Meets standard.	
  \item[\bf 4:] The purpose is nearly there, but is missing something important
  \item[\bf 3:] The purpose is not accurate or too vague.
\end{itemize}


\paragraph{Procedure} It is easy for the reader to get an idea of what you did during the experiment and to follow the calculations.
\begin{itemize}
  \item[\bf 10:] Exceeds Standard. You have to go above and beyond to get a perfect score here!
  \item[\bf 8:] Meets standard.
  \item[\bf 6:] Your procedure lacks detail or it is unclear where your algebraic calculations came from.
  \item[\bf 4:] Does not meet standard.  Although this section exists, there is not nearly enough detail given.
\end{itemize}

\paragraph{Data} All relevant measurements are provided and well-labeled.  Changes made at each trial are explained.	
\begin{itemize}
  \item[\bf 10:] Meets standard.	
  \item[\bf 8:] Even though you have included all of the measurements, it is hard to tell what they represent. Labeling or explanations could have been better. 
  \item[\bf 6:] Your data exists, but it is unclear what measurements are or where they came from.
  \item[\bf 4:] Does not meet standard.  Although this section exists, there is not nearly enough detail given.
\end{itemize}

\paragraph{Conclusion} Your conclusion is complete and answers all of the questions.  It thoroughly covers what you learned and what the errors might have been	
\begin{itemize}
  \item[\bf 5:] Meets standard.	
  \item[\bf 3:] Your discussion of errors is vague or your discussion of what you learned does not address the overall purpose of doing the lab.
  \item[\bf 2:]Does not meet standard. Although this section exists, there is not nearly enough detail given.
\end{itemize}




\end{document}