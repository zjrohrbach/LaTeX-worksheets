\documentclass[12pt]{exam}
\usepackage[phy]{template-for-exam}
\usepackage{tikz,ifthen,multicol,siunitx}
\usepackage[margin=0.5in]{geometry}
\footer{}{}{}
\header{}{}{}
\shadedsolutions
\printanswers
\usetikzlibrary{
  calc,
  patterns,
  decorations.pathmorphing,
  decorations.markings,
  arrows,
  shapes,
  positioning,
  math,
  intersections,
  fadings
}

\def\mystrut{\protect\rule[-2.2ex]{0ex}{2.2ex}} 
\qformat{ \textbf{Task \#\thequestion}
  \ifthenelse{\equal{\thequestion}{\thequestiontitle}}
    {}
    {: \emph{\thequestiontitle}}
  \mystrut  \hfill}

\begin{document}

\Large

  \tikzstyle{directed}=
    [postaction={decorate,decoration={markings,
    mark=at position .65 with {\arrow[scale=1.5]{stealth}}}}]
  \tikzstyle{reverse directed}=
    [postaction={decorate,decoration={markings,
    mark=at position .5 with {\arrowreversed[scale=1.5]{stealth}}}}]


\begin{questions}


\question
  The speed of light in acryllic is \SI{2.01e8}{m/s}. 
  
  \begin{parts}
    \part  
      Calculate the angle of refraction as the ray of light below enters the acryllic block shown below.
    \part 
      Calculate the angle of refraction as the ray of light exits the acryllic block.
  \end{parts}

  \vspace{2em}

  \begin{tikzpicture}[]
    \def\angle{120}
    \draw[fill=blue!30] (-5,0) rectangle (5,-4);
    \draw[reverse directed,orange,very thick] 
      (0,0) -- (\angle:4);
    \draw[dashed] (0,-1) -- (0,3);
    \draw (\angle:1.5) arc (\angle:90:1.5) node[above,midway] {$30^\circ$};

  \end{tikzpicture}




\vs \hrule \vs

\question
  You are under water and are shining a laser upward through the surface of the water.  The incident angle of the laser on the surface is $62^\circ$.  What will be the angle of refraction? (The index of refraction for water is 1.33).

\end{questions}
 
\vs \hrule \vs

Task \#1 Answer

$n=1.49$; $\theta_2=19.6^\circ$; $\theta_3=30^\circ$


\vspace{1em}

Task \#2 Answer

undefined; $\theta_c=48.8^\circ$



\end{document}