\documentclass[12pt]{exam}
\usepackage[phy]{template-for-exam}
\usepackage{tikz,ifthen,multicol,siunitx}
\footer{}{}{}
\header{}{}{}
\shadedsolutions
\printanswers
\usetikzlibrary{shadings,decorations.pathmorphing,arrows.meta,patterns}

\def\mystrut{\protect\rule[-2.2ex]{0ex}{2.2ex}} 
\qformat{ \textbf{Task \#\thequestion}
  \ifthenelse{\equal{\thequestion}{\thequestiontitle}}
    {}
    {: \emph{\thequestiontitle}}
  \mystrut  \hfill}

\begin{document}

\vspace*{\stretch{1}}


\begin{questions}

\Large

\question
The radio station WZPL broadcasts at a frequency of 99.5 MHz.  Radio waves travel at a speed of \SI{3.00e8}{\meter\per\second}.  (\emph{Hint:} $\SI{1}{\mega\hertz}=\SI{e6}{\hertz}$)

\begin{parts}
  \part How long is the wavelength of WZPL's radio waves? 
  \part What is the period of these waves?
\end{parts}




\vs \hrule \vs

\question
A certain string has a length of 3.2 meters.  Its fourth harmonic occurs at 112 Hz.

\begin{parts}
  \part What is the wavelength?
  \part How far apart are the nodes?
  \part What is the fundamental frequency of the string?
  \part What is the wave speed through the string?
\end{parts}

\vspace*{\stretch{1}}
  
\end{questions}








\end{document}