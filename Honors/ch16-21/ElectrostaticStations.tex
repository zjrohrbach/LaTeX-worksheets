\documentclass[10pt]{exam}
\usepackage[hon]{template-for-exam}
\usepackage{silence,tikz,multicol,hyperref}
\WarningFilter{latex}{Label `question}
\WarningFilter{latex}{There were multiply-defined labels}

\title{Electrostatics Stations}
\author{Rohrbach}
\date{\today}

\begin{document}
\maketitle

\section{Section 16-1 and Charged Rods}

\begin{questions}
  \question
    Read the first two paragraphs in 16-1 about the experiment.  Try to reproduce the results by using the rubbing the cloth on the plastic and glass rods. 

    \begin{parts}
      \part 
        Rub the suspended piece of tubing and the white wand.  Determine whether they attract or repel.
      \part
        Do the same for the grey vinyl wand.
      \part
        Can you draw any conclusions? \vs
    \end{parts}

  \question
    Read the rest of Section 16-1.  What is \emph{charge}? \vs

  \question
    What is the Law of \emph{Conservation of Electric Charge}? \vs
   
\end{questions}

\section{Section 16-2 and the Balloon PhET Lab}

\begin{questions}
  \question
    Go to \href{https://tinyurl.com/BalloonPhysics}{\texttt{https://tinyurl.com/BalloonPhysics}} (case sensitive). Looking at the lab, what are the two types of electric charges? \vs

  \question
    Try rubbing the balloon on the sweater so that the balloon and sweater have opposite net charges.  Do they seem to attract or repel each other? \vs

  \question
    Reset and now choose two balloons.  Rub each balloon equally on the sweater so they are equally negatively charged.  Do the balloons seem to attract or repel one another? \vs

  \question
    Which type of charge can move? \vs

  \question
    Read Section 16-2 to explain why only this type of charge can move. \vs

\end{questions}

\pagebreak

\section{Section 16-3 and the Wimshurst Machine}

\begin{questions}
  \question
    Place the two rods and inch or two apart from each other and steadily turn the handle until sparks jump between the two rods.  Why do you think this happens? \vs

  \question 
    Use your book (Section 16-3) to explain the difference between conductors and insulators\vs 
    
  \question 
    Move the rods on the Wimshurst Machine further away and place the glass beaker between them.  Do the same pair of pliers. (\emph{note: Please do not use the pliers to grab the machine.  You are just using the metal ends of the pliers.})  Based on your observations, which one is an insulator, which is a conductor, and why do you come to that conclusion? \vs 

  \question 
    Why is it helpful to hold the pliers from the rubber end? \vs 

  \question 
    According to the book, what is different about the atoms of insulators and conductors that make them have different conductive abilities?\vs 

\end{questions}

\section{Section 16-4 (first part)} \label{16-4}

\begin{questions}
  \question 
    Read the first part of Section 16-4 (stop when you get to the post-it note).  What is charging by contact? \vs

  \question 
    What does it mean if charge is induced in a conductor? \vs
  
  \question 
    What is grounding and why is it important? \vs
  
  \question 
    How can you use grounding to charge an object by induction? \vs
  
  \question 
    Can charge separation be achieved in insulators?  Explain. \vs
    
\end{questions}

\pagebreak

\section{Section 16-4 (second part) and the Electroscopes}

\begin{questions}
  \question
    Use a piece of cloth to charge the plastic rod.  Then bring it close to the electroscope bulb.  What happens? \vs

  \question
    Touch it to the bulb.  What happens now? \vs

  \question
    Now, read 16-4 and explain why it's happening. \vs

  \question
    Try to recreate the effect shown in Figure 6-12.  Charged glass rods are positive.  Charged plastic rods are negative. \vs
    
\end{questions}

\section{Van De Graff Generator}

\begin{questions}
  \question
    Have one person stand on the stool, and put their hands on the generator.  The generator should make the person feel as if their hair is standing on end.  Why do you think this is? \vs

  \question
    Have the person on the stool touch someone else in the group (warning, you will get a little shock).  Why do you think the person who was touched was shocked even though the person constantly touching the Van De Graf is not continuously being shocked? \vs

\end{questions}

\pagebreak

\section{Section 16-5 (first part)}

Use Section 16-5 to answer the following questions:

\begin{questions}
  \question
     What is the Coulomb's law equation and what do each of the symbols stand for? \vs[2]

  \question
    What is the unit for measuring charge? \vs
  
  \question
    What does $e$ mean? \vs
  
  \question
    What does it mean to say electric charge is \emph{quantized}? \vs
  
  \question
    What are similarities and differences between Coulomb's Law and the Law of Universal Gravitation? \vs  
    
\end{questions}

\section{Section 16-5 (second part)}

\begin{questions}
  \question
    What is \emph{electrostatics}? \vs

  \question
    Read through Example 16-1 and try to understand it.  Then work this problem:

    Two charges of magnitude \SI{+5}{\micro\coulomb} and \SI{-2}{\micro\coulomb} are placed \SI{2}{\milli\meter} apart from each other.  Calculate the magnitude of the force between them.  Remember, $\SI{1}{\micro\coulomb}=\SI{e-6}{\coulomb}$ and $\SI{1}{\milli\meter}=\SI{e-3}{\meter}$.  \emph{Check your answer below.} \vs[3]

    {\footnotesize Answer: \SI{22500}{\newton}}
    
\end{questions}

%\pagebreak

% \section{John Travoltage}

% \begin{questions}
%   \question 
%     Go to \href{https://www.tinyurl.com/travoltage}{\texttt{https://www.tinyurl.com/travoltage}}.  Place John Travoltage's finger so he is not in direct contact with the door knob, but relatively close.  Rub his foot on the carpet.  What happens?  Why do you think this is? \vspace{6em}
    
%   \question 
%     From where is John Travoltage getting his extra charge, and what charge do you think this is (positive or negative)? \vspace{6em}

%   \question
%     For both the transfer of charge from the carpet to his foot, and from his hand to the doorknob, decide if they are transfer by contact or induction, and explain your reasoning. (\emph{If you haven't done Station~\ref{16-4} yet, you'll need to come back to this question after you've learned about charging by contact and induction.}) \vspace{6em}
    

    
% \end{questions}



\end{document}