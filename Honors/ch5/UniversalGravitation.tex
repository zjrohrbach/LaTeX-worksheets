\documentclass[10pt]{exam}
\usepackage[hon]{template-for-exam}
\usepackage{tikz,graphicx,xcolor}

\title{Newton's Law of Universal Gravitation.}
\author{Rohrbach}
\date{\today}

\begin{document}
\maketitle

\section*{The Newtonian Synthesis}

\definecolor{mygreen}{HTML}{669c35}

\begin{tikzpicture}
  \fill[rounded corners,mygreen,draw=black] 
    (0,2.4) -- (-0.5,0) -- (0.5,0) -- cycle;
  \node at (0,0) {\includegraphics[height=4cm,angle=-50]{earth.png}};
\end{tikzpicture}

\vs

\section*{The Law}

\begin{itemize}
  \item all objects with mass \fillin[attract][10em] all other objects in the universe with a force called gravity.
  \item gravity is \fillin[directly proportional][15em] to the product of the two masses.
  \item gravity is \fillin[inversely proportional][15em] to the square of the separation between the two objects.
  
  \begin{center}
    \begin{tikzpicture}
      \draw (0,0) rectangle (8,2);
    \end{tikzpicture}
  \end{center}


  \item $G$ is the \fillin[universal gravitational constant][20em], which is a measure of the strength of gravity
  
  \begin{center}
    \begin{tikzpicture}
      \draw (0,0) rectangle (12,2);
    \end{tikzpicture}
  \end{center}

\end{itemize}

\pagebreak

\section*{Examples}

\begin{questions}
  \question 
    Calculate the force of gravity between a 5-kg bowling ball and a 0.4-kg paperweight that are separated by 20~cm.
    \vs


  \question 
    Calculate the \emph{acceleration of gravity} on the surface of the Earth.
    \vs
\end{questions}




\end{document}