\documentclass[10pt]{exam}
\usepackage[phy]{template-for-exam}

\title{Energy \#1}
\author{Rohrbach}
\date{\today}

\begin{document}
\maketitle

\begin{questions}

\question
  A block is pushed with force of 100 N over a displacement of 2.3 m.

  \begin{parts}
    \part
      What is the work done on the block?
      \vs

    \part
      If it took 0.9 s to do this, what is the power output on the block?
      \vs

  \end{parts}

\question
  It takes \SI{50000}{\joule} of work to push a car \SI{1371}{\meter}.  What force is required?
  \vs 

\question
  A 100 W light bulb is left on for 33 s.  How much work does it do?
  \vs 


\pagebreak

\question
  A 3.75-kg watermelon is lifted 2.4 m into the air: 

  \begin{parts}
    \part
      Calculate the force that you are doing work against.
      \vs

    \part
      How much work is needed to lift the watermelon that high (think about your answer to a to help with this answer)?
      \vs

    \part
      If it took you 5.1 s to lift it 2.4 m, what is your power?
      \vs

  \end{parts}


\question 
A cable lifts a 2600-kg elevator to a height of 300 m in a time of 8 seconds.  What power does the motor produce in accomplishing this task? (\emph{Hint:} You will need to find the work done first.  Use problem 4 to help you as they are similar problems)
\vs[2]



  
\end{questions}

\end{document}