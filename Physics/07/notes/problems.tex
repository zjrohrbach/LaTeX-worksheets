\documentclass[12pt]{exam}
\usepackage[
  paperheight=2.125in, 
  paperwidth=5.5in,
  margin=.3in
]{geometry}
\usepackage{siunitx}

\begin{document}
\pagestyle{empty}

\paragraph{Ex 1)}
  Starting from rest, a child zooms down a frictionless slide from an initial height of 3.00 m.  What is her speed at the bottom of the slide?  Assume she has a mass of 25.0 kg.

\pagebreak

\paragraph{Ex 2)}
  You slide a trashcan ($m = \SI{10.2}{\kilo\gram}$) across the floor with an initial velocity of 7.9 m/s.
  
  \vspace{1em}
  
  \begin{parts}
    \part
      If the trashcan eventually stops, what is the work done by friction?

    \part
      If the force of friction is \SI{-29.4}{\newton}, how far does the trash can go?
    \end{parts}


\paragraph{Ex 3)}
  Robert ($m = \SI{75}{\kilo\gram}$-) starts at rest at the top of a carnival slide, which is 20 m above the ground.  As he slides down, friction does \SI{1500}{\joule} of work.  How fast is he going when he gets to the ground?


\end{document}