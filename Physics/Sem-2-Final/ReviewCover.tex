\documentclass[10pt]{exam}
\usepackage[phy]{template-for-exam}

\title{Physics I - Spring Final Exam Review}
\author{Rohrbach}
\date{\today}

\begin{document}
\maketitle

\section*{Minimum Requirements}

You must answer all the multiple choice questions.  You must also pick 10 problems to do on a separate sheet of paper.  If you do more than 10 problems, it will be bonus (see below).  This will be taken for a 15-point grade on exam day.  Please note: if you show no work for any problem, the maximum score you can get on this assignment is a 7/15.

\section*{Extra Credit}

Extra credit will be applied to your exam grade.  The exam is about 45 questions at 2 points per question, so each point of extra credit counts as about a 1\% bump on your exam grade.

\begin{itemize}
  \item You will get a 1 point for not skipping any problems.
	\item You may explain your answers for each question.  For every 10 questions you explain, you get 1 point.  (Do this on a separate sheet of paper)
	\item You may name every variable in each equation on the equation sheet and give the units for each to get an additional 1 point
	\item You may define all of the ``important terms'' for each section for 1 point. (Do this on a separate sheet of paper)
	\item Maximum extra credit is 10 points (upwards of a 10\% bump to your exam grade)
	\item Any extra credit you do must be hand written.  It may NOT be typed!
\end{itemize}

\section*{The Final Exam}

The final exam will consist of 40-50 multiple choice questions.  (Problems will be included, but they will be answered as multiple choice).

\section*{Online Access to the Textbook}

Feel free to check out a textbook for study or access it online:  Go to \texttt{www.masteringphysics.com} and use one of the following (case sensitive) username/password combinations:

\vspace{1em}

\begin{tabular}{lp{8em}|lp{8em}|lp{8em}}
  Username: & \texttt{avon-physics-1} &
  Username: & \texttt{avon-physics-2} &
  Username: & \texttt{avon-physics-3} \\
  Password: & \texttt{ScienceRulez1}  &
  Password: & \texttt{ScienceRulez1}  &
  Password: & \texttt{ScienceRulez1} 
\end{tabular}

\end{document}