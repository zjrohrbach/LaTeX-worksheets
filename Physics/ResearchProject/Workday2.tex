\documentclass[10pt]{exam}
\usepackage[phy]{template-for-exam}
\usepackage{hyperref,enumitem,multicol}



\title{Research Project - Workday \#2}
\author{Rohrbach}
\date{\today}

\begin{document}
\maketitle

\noindent
\emph{Today is the last in-class workday on your project.  You still have nearly a month to work on it, but do not expect any more extended periods of in-class work time.}

\section*{Presentation Date}

Write down the day that you are scheduled to give your presentation here: \fillin[][15em]

\section*{Project Checklist}

\begin{checkboxes}

  \choice 
    Have I turned my Annotated Bibliography and made corrections to get my 20/20?  Remember, your grade is frozen after Friday.

  \choice
    Do I know my topic well enough that I can explain it in my own words?

  \choice
    Do I know my topic well enough that I can answers questions about it?

  \choice
    Does my project thoroughly cover the physics of my topic (not just the history or the biography)?

  \choice
    Is my presentation 5-7 minutes?

  \choice
    Do I have both a \emph{visual aid} and a \emph{presentation aid}?

  \choice
    Are all my sources cited in my presentation aid?  Even any new sources I've used?  Remember, all sources need to be cited (although not annotated.)

\end{checkboxes}

\section*{Visual Aid}

\begin{itemize}
  \item 
    You are to have one visual aid \emph{which substantially contributes to understanding your topic}. (For example, a picture of Albert Einstein does not substantially contribute to our understanding of Einstein.  A diagram of how the photoelectric effect---something he is famous for describing---works does substantially contribute.)  A visual aid is distinct from your \emph{presentation aid}.
  
  \item 
    Examples of visuals include: a 3D model/diaramma, a graph, a flow chart, an experiment, a	diagram, a model, a comic book, a pop-up book, or a skit.

  \item
    Your visual should be something you've made, not just a Google Image.  It can be based off of something you've found, but you should do something to make it your own.

  \item  
    If you use or are inspired by any visuals you have found from another source, \emph{you need to make sure that source gets cited.}

  \item
    You need to be able to explain your visual.
  
\end{itemize}


\section*{Helpful Links}

All materials that you have been given are available on Schoology in the ``{\bf Research Project Information}'' folder.  Quick Links to some of the highlights are provided below for convenience:

\begin{itemize}
  \item 
    Research Pathfinder: \texttt{\href{https://go.rohrbachscience.com/pathfinder}{https://go.rohrbachscience.com/pathfinder}}
  \item 
    List of Topic Assignments: \texttt{\href{https://go.rohrbachscience.com/project-topics}{https://go.rohrbachscience.com/project-topics}}
  \item 
    Turn in Annotated Bibliography: \texttt{\href{https://go.rohrbachscience.com/annotated-bib}{https://go.rohrbachscience.com/annotated-bib}}
  \item 
    Project Assignment \& Rubric: \texttt{\href{https://go.rohrbachscience.com/project-assignment}{https://go.rohrbachscience.com/project-assignment}}
\end{itemize}


\end{document}