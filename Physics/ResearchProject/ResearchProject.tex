\documentclass[11pt]{exam}
\usepackage[phy]{template-for-exam}
\usepackage{multicol,enumitem,hyperref,qrcode,floatflt}

\title{Research Project}
\author{Rohrbach}
\date{\today}

\begin{document}
\maketitle

\section*{Purpose}
 You will prepare a {\bf 5- to 7-minute presentation} to teach your classmates a physics concept.  

\section*{Topic}



For this project, you must choose a topic that you want to learn more about.  The topic can be a physics topic that we have not yet covered in detailed, or it can be a real-world application of a topic that we have covered.

\begin{floatingfigure}{1.1in}
  \qrcode{https://go.rohrbachscience.com/project-topics}
\end{floatingfigure}

You can find a list of available topics by using the QR code at the right or by navigating to \texttt{\href{https://go.rohrbachscience.com/project-topics}{go.rohrbachscience.com/project-topics}}.  If there is a topic you are interested in that is not on the sheet, please check with Mr. Rohrbach to have it approved.






\section*{Research}
We will spend some time in the library working on this project.  Both book and internet sources are valid.  You must have at least 3 sources of information.  {\sc Wikipedia is not a valid source for this project.} 

\section*{Presentation}
The core of your presentation will be a visual that you design.  This can be an image, a graph, a photo, a video, or a model.  Your visual should substantially contribute to understanding the topic.  Pretty pictures are welcome to be included, but pretty pictures alone do not count for this requirement. (For example, a picture of a person does not count.) 

Examples of visuals might include: 

\begin{multicols}{3}
  \begin{itemize}
    \item	graph
    \item	flow chart
    \item	collage
    \item	experiment
    \item	large drawing
    \item	model
    \item	interpretive dance
    \item	pop-up book
    \item	skit
  \end{itemize}
\end{multicols}
 

In addition to your visual, you will likely have a PowerPoint, poster, or other presentation aid.  This is not necessarily a requirement.  However, it is often helpful and highly encouraged to have a presentation aid, especially if your visual alone cannot sustain a 5- to 7-minute presentation.  

Your presentation must include specifics as to the science of your topic (for example: equations, explanations, scientists who helped figure it out, history, benefits to society).  \emph{The explanation of the science should be in your own words, not just copied down out of a book.}  I want you to understand a little of what's going on.  

\pagebreak

\section*{On the Grading Rubric}
This is an extended project, and every year I have students who think my grading rubric is a little harsher than they are used to seeing.  As such, I think it is important to summarize what differentiates A-level, B-level, and C-level projects.

  \begin{itemize}
    \item {\bf C projects} are projects that generally meet the requirements.  Usually they are a simple power point or poster.  The student explains the topic well enough, but the explanation does not go very far in depth.  A C-project gets the idea across, but it lead me to learning more about the topic than I could have done by simply reading one of your sources.
    \item {\bf B projects} are good, solid projects.  Students are well aware of what they are talking about and can answer my questions.  A well-done poster or power point is probably a B project.  You should be proud of a B project!
    \item {\bf A projects} go above and beyond.  There is something about the project that makes it stick out: either a creative flair, a passion in how the student presents it, or a description that makes me think about something in a way I never have before
    \item {\bf A+ projects} are projects that I could not imagine having been done any better!
  \end{itemize}

If you look at the rubric, you will realize that if you meet all standards, you will get a 90\%.  Does that mean it is impossible to get a 100\%?  No, of course not.  But perfect projects go above and beyond in multiple respects: they are remarkably well researched, they are very creative (something I have either not seen before or not seen completed nearly as well), and they really stick in my mind as outstanding projects.  If you want a 100\% you will need to work very hard and have a fantastic idea that you learn inside and out!!!

\section*{On Plagiarism}
What is plagiarism? If ever you are copy and pasting something from a website, you better indicate the website where this material came from and it better be in quotation marks. However, even if it is cited, a presentation still counts as plagiarism there is too much use of others' words and not enough use of your own. \emph{This is not an exhaustive list.} You are responsible at this point in your high school career for knowing what is and is not plagiarism. \emph{If you are unsure whether or not you are plagiarizing, talk to Rohrbach.}

\section*{Timeline}

\begin{tabular}{ll}
  Mon, Jan 15 & 
  Sign up for your top 3 topic choices
  \\[1em]
  Tue-Wed, Jan, 23-24 & 
  Library Research Day \#1
  \\[1em]
  Sun, Jan 28 & 
  Annotated Citations due on Schoology by 11:59pm
  \\[1em]
  Wed-Thu, Feb 14-15 & 
  Library Research Day \#2
  \\[1em]
  Week of Mar 11-15 & 
  Presentations in class
  \\[1em]
  Sat, Mar 16 &
  Spring Break!
\end{tabular}


\pagebreak

\setlist[itemize]{
    topsep=0pt,
    itemsep=-1ex,
    partopsep=1ex,
    parsep=1ex,
    label = $\bigcirc$
  }


\section*{Rubric}

\paragraph{Creativity (15 points):} 
  Your project is not just a generic ``book report''-style presentation.  There is some level of creativity to it.

\begin{itemize}
  \item Exceeds Standard (15/15)
  \item Meets Standard (13/15)
  \item Almost Meets Standard (11/15)
  \item Needs Work (7/15)
  
\end{itemize}




\paragraph{Effectiveness and Self Sufficiency (35 points):} 
  There is evidence that you know and understand what you are talking about and are not merely parroting something you read.  (\emph{If you are reading off of slides and are not able to answer questions about your project, your explanation is not self-sufficient.})

\begin{itemize}
  \item Exceeds Standard (35/35)
  \item Meets Standard (32/35)
  \item Almost Meets Standard (25/35) - 
    Although the words are your own, your presentation seems more like a paraphrase of your sources than it seems like your own presentation. 
  \item Needs Work (18/35) -
    You have not done enough to make the explanation ``your own.''
  
\end{itemize}



\paragraph{Scientific Accuracy and Thoroughness (30 points):} 
  You thoroughly and accurately explain the \emph{**science**} (not just the history) behind your topic.
  

\begin{itemize}
  \item Exceeds Standard (35/35)
  \item Meets Standard (32/35)
  \item Almost Meets Standard (25/35) - 
    Perhaps you could have gone into more depth.  Perhaps everything you said was pretty good, but the organization and way you said it was a little hard to follow.  Perhaps your explanation was mostly right, but showed a few misunderstandings in how your topic worked.  Perhaps you weren't as prepared for the Q\&A session as you could have been.
  \item Needs Work (18/35) -
    Perhaps you focused more on historical and biographical details than on the science.
  
\end{itemize}



\paragraph{Your Visual [Image/Diagram/Picture/Model] (15 points): } 
  Your visual substantially contributes to understanding of the topic \emph{and} is well explained. (A picture of a person does not count for this standard.) 
  

\begin{itemize}
  \item Exceeds Standard (15/15)
  \item Meets Standard (13/15)
  \item Almost Meets Standard (11/15) - 
    Perhaps you have a very well done graphic but it could use some more explanation.  Perhaps it is explained well but could use some more detail in the graphic itself.
  \item Needs Work (9/15) -
    Your visual is good, but it has just been copied and pasted from the internet
  \item Visual exists but does not meet standard (7/15)
  \item No visual is provided, except maybe some generic pictures (0/15)
  
\end{itemize}

\paragraph{Total Score:}
  \fillin[][3em] / 100
 


\end{document}