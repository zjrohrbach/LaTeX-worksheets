\documentclass[11pt]{exam}
\usepackage[phy]{template-for-exam}
\usepackage{multicol}

\title{Research Project}
\author{Rohrbach}
\date{\today}

\begin{document}
\maketitle

\section*{Purpose}
To connect what we've learned this year to real-world applications or to learn about topics we did not cover.

\section*{Topic}
For this project, you must choose a topic.  This topic should be something that you want to learn more about, or something in which you are interested.  It will focus around something that we have not yet covered in detail.  A topic signup sheet will be available in class.  If there is a topic you are interested in that is not on the sheet, please check with and have it approved by Rohrbach.

\section*{Research}
We will spend some time in the library working on this project.  Both book and internet sources are valid.  You must have at least 3 sources of information.  {\sc Wikipedia is not a valid source for this project.}  You may look up ideas there, but you may not cite it.

\section*{Presentation}
The core of your presentation will be a visual that you design.  This can be an image, a graph, a photo, a video, or a model.  Your visual should substantially contribute to understanding the topic.  Pretty pictures are welcome to be included, but pretty pictures alone do not count for this requirement. (For example, a picture of a person does not count.) 

Examples of visuals might include: 

\begin{multicols}{3}
  \begin{itemize}
    \item	graph
    \item	flow chart
    \item	collage
    \item	experiment
    \item	large drawing
    \item	model
    \item	interpretive dance
    \item	pop-up book
    \item	skit
  \end{itemize}
\end{multicols}
 

In addition to your visual, you will likely have a PowerPoint, poster, or other presentation aid.  This is not necessarily a requirement.  However, it is often helpful and highly encouraged to have a presentation aid, especially if your visual alone cannot sustain a 5- to 7-minute presentation.  

Your presentation must include specifics as to the science of your topic (for example: equations, explanations, scientists who helped figure it out, history, benefits to society).  \emph{The explanation of the science should be in your own words, not just copied down out of a book.}  I want you to understand a little of what's going on.  

\pagebreak

\section*{On the Grading Rubric}
This is an extended project, and every year I have students who think my grading rubric is a little harsher than they are used to seeing.  As such, I think it is important to summarize what differentiates A-level, B-level, and C-level projects.

  \begin{itemize}
    \item {\bf C projects} are projects that generally meet the requirements.  Usually they are a simple power point or poster.  The student explains the topic well enough, but the explanation does not go very far in depth.  A C-project gets the idea across, but it lead me to learning more about the topic than I could have done by simply reading one of your sources.
    \item {\bf B projects} are good, solid projects.  Students are well aware of what they are talking about and can answer my questions.  A well-done poster or power point is probably a B project.  You should be proud of a B project!
    \item {\bf A projects} go above and beyond.  There is something about the project that makes it stick out: either a creative flair, a passion in how the student presents it, or a description that makes me think about something in a way I never have before
    \item {\bf A+ projects} are projects that I could not imagine having been done any better!
  \end{itemize}

If you look at the rubric, you will realize that if you meet all standards, you will get a 91\%.  Does that mean it is impossible to get a 100\%?  No, of course not.  But perfect projects go above and beyond in multiple respects: they are remarkably well researched, they are very creative (something I have either not seen before or not seen completed nearly as well), and they really stick in my mind as outstanding projects.  If you want a 100\% you will need to work very hard and have a fantastic idea that you learn inside and out!!!

\section*{On Plagiarism}
What is plagiarism? If ever you are copy and pasting something from a website, you better indicate the website where this material came from and it better be in quotation marks. However, even if it is cited, a presentation still counts as plagiarism there is too much use of others' words and not enough use of your own. \emph{This is not an exhaustive list.} You are responsible at this point in your high school career for knowing what is and is not plagiarism. \emph{If you are unsure whether or not you are plagiarizing, talk to Rohrbach.}

\section*{Timeline}

\begin{tabular}{ll}
  Mon, Jan 9 & 
  Indicate your top 3 topic choices
  \\[1em]
  Wed-Thu, Jan, 25-26 & 
  Library Research Day \#1
  \\[1em]
  Sun, Jan 29 & 
  Annotated Citations due on Schoology by 11:59pm
  \\[1em]
  Wed-Thu, Feb 22-23 & 
  Library Research Day \#2
  \\[1em]
  Week of Mar 13-16 & 
  Presentations in class
  \\[1em]
  Sat, Mar 18 &
  Spring Break!
\end{tabular}
 


\end{document}