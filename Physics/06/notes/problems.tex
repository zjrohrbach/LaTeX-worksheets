\documentclass[12pt]{article}
\usepackage[
  paperheight=2.125in, 
  paperwidth=5.5in,
  margin=.3in
]{geometry}
\usepackage{tikz,graphicx}
\usetikzlibrary{shadings,decorations.pathmorphing}

\tikzstyle{vector}=[
  ->,
  red,
  thick
]


\begin{document}
\pagestyle{empty}

\paragraph{Ex \#1) A (Mostly) \emph{Elastic} Collision} 

A 5-kg bowling ball is rolling at a speed of 10 m/s toward a stationary 2-kg soccer ball.  After the collision, the bowling ball is still moving at 4.1 m/s.  How fast is the soccer ball moving?

\pagebreak

\paragraph{Ex \#2) A Totally \emph{Inelastic} Collision} 

A sports car with a mass of 1000 kg is traveling {\bf east} with a velocity of 45 m/s. It decides to play chicken with a semi-truck with a mass of 9000 kg and a velocity of 14 m/s to the {\bf west}. After impact, the vehicles stick together. What is the final velocity of the combined vehicles?

\pagebreak


\paragraph{Ex \#3) Recoil} 

A 1500-kg cannon fires a 8.3-kg cannonball at a speed of 32 m/s.  How fast does the cannon recoil?




\end{document}