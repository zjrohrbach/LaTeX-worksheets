\documentclass[10pt]{exam}
\usepackage[phy]{template-for-exam}

\title{Inelastic Collisions Lab}
\author{Rohrbach}
\date{\today}

\begin{document}
\maketitle

\paragraph{Purpose:} To use conservation of momentum to calculate how far out a marble will land from the table.

\begin{questions}

\question \emph{Setup:} Place a piece of tape on the back of the wooden ring and clamp the ramp onto the edge of the table so that the black bolt is just hanging over the edge

\question	
Measure the mass of both the small marble and the large wooden ring (with tape)

\begin{center}
  marble: \fillin[][8em]
  \hspace{5em}
  wooden ring: \fillin[][8em]
\end{center}


\question
Place a photogate at the bottom of the ramp just as the marble would leave the table. Do five trials and then find the average velocity the marble has when it leaves the table.

\begin{center}
  \begin{tabular}{*5{|p{5em}}||p{5em}|}
    \hline
    \multicolumn{6}{|c|}{Velocity (m/s)}\\
    \hline
    Trial \#1 &	
    Trial \#2	&
    Trial \#3	&
    Trial \#4	&
    Trial \#5 &
    Average   \\
    \hline
    &&&&& \\[1em]
    \hline
  \end{tabular}
\end{center}
					

\question \label{predictmarble}
Calculate how far the marble should travel out before reaching the ground if it takes 0.39 s to reach the ground. (\emph{Hint:} use $v = d/t$).
\vs

\question
Place a meter stick at the bottom of the table, roll the marble off the edge and measure to see where it lands. Use the percent error equation to figure out how accurate your numbers are.
%
\begin{align*}
  \text{\% error} &=  
  \frac{
    \left|\text{measured}-\text{expected}\right|
    }{
      \text{expected}
    } 
  \times 100
\end{align*}

\begin{center}
  Measured: \fillin[][8em] meters
  \hspace{4em}
  Expected (from \#\ref{predictmarble}): \fillin[][8em]  meters
  \vspace{2em}
\end{center}


\begin{flushright}
  Percent Error: \fillin[][5em]\%
\end{flushright}

\pagebreak

\begin{EnvUplevel}
  For your trial, you are going to place the wooden ring on the end of the black bolt, with the tape facing away from where the marble will roll. You are going to roll the marble down the ramp and into the ring, but {\bf don't do that until after you have finished question \#\ref{predictinelastic}}.
\end{EnvUplevel}


\question
Use conservation of momentum to calculate how fast the marble and ring will move together assuming they stick together after the collision (not a perfect assumption, but pretty close).
\vs[3]

\begin{flushright}
  $v =$ \fillin[][8em] m/s
\end{flushright}


\question \label{predictinelastic}
Now use $v = d/t$ to calculate how far out from the table the marble should land. Remember, the time to fall is still 0.39 s.
\vs

\begin{flushright}
  $d =$ \fillin[][8em] m
\end{flushright}

\question
Place a meter stick at the bottom of the table, roll the marble off the edge and measure to see where it lands. Use the percent error equation to figure out how accurate your numbers are.
%
\begin{align*}
  \text{\% error} &=  
  \frac{
    \left|\text{measured}-\text{expected}\right|
    }{
      \text{expected}
    } 
  \times 100
\end{align*}

\begin{center}
  Measured: \fillin[][8em] meters
  \hspace{4em}
  Expected (from \#\ref{predictinelastic}): \fillin[][8em]  meters
  \vspace{2em}
\end{center}

\begin{flushright}
  Percent Error: \fillin[][5em] \%
\end{flushright}

\question
Use momentum to explain why the marble did not travel out as far after colliding with the ring as it did without the collision.
\vs

\question
Discuss your percent errors. Do you consider this experiment accurate? Why? What are some sources of error in this lab?
\vs


\end{questions}

\end{document}