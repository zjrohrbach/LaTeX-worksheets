\documentclass[10pt]{exam}
\usepackage[phy]{template-for-exam}
\usepackage{my-tikz-clipart}

\title{Newton \#2}
\author{Rohrbach}
\date{\today}




\begin{document}
\maketitle

\begin{questions}
  
  \question
    A cat sits at rest on the top of a table.
  
    \begin{parts}
      
      \part
        Is the cat at mechanical equilibrium?  How do you 
        know?
        \vs

      \part
        Draw the free body diagram.  
        
        Make sure to include the net force.
        
        \begin{center}
          \begin{tikzpicture}
            \draw[pattern=north east lines]
              (-2,.1) rectangle (2,-.1);
            \cat[contour=blue,scale=0.7]
          \end{tikzpicture}
        \end{center}
        \vspace{1em}


      \part
        If the cat's weight (that is, force of gravity) is 90 Newtons, what is the normal force acting on the cat?
        \vs

    \end{parts}

  \hrule

  \question
    A car is moving east along the interstate {\bf at a 
    constant speed}.

    \begin{parts}
  
      \part
        Is the car at mechanical equilibrium?  How do you 
        know?
        \vs
    
      \part
        Draw the free body diagram.  
        
        Make sure to include the net force.
        
        \cardrawing
        \vspace{2em}
      
      \part
        The car has an applied force of 700 Newtons and a 
        frictional force of 300 Newtons.  Find the 
        magnitude of the air resistance.
        \vs

    \end{parts}


  \hrule

  \question
    The same car is now {\bf accelerating forward}.

    \begin{parts}
      
      \part
        Is the car at mechanical equilibrium?  How do you 
        know?
        \vs
    
      \part
        Draw the free body diagram.  
        
        Make sure to include the net force.
        
        \cardrawing
        \vspace{2em}

    
      \part
        The frictional force and air resistance are the 
        same as in the last problem, but this time, the 
        car has an applied force of 900 Newtons.  
        Calculate the net force.
        \vs

    \end{parts}


  \pagebreak

  \question
    You are lifting a bucket with a rope.  The force on 
    the rope is 45 Newtons and the bucket has a weight
    (that is, force of gravity) of 23 Newtons.
    
    \begin{parts}

      \part
        Draw the free body diagram.  
        
        Make sure to include the net force.

        \begin{center}
          \begin{tikzpicture}
            \path (0,-1.8) pic {bucket};
            \filldraw[pattern=north east lines] 
            (-.1,0) -- (.1,0) -- (.1,2) -- (-.1,2) -- cycle;
          \end{tikzpicture}
        \end{center}
        \vs
        
      \part
        Calculate the net force on the bucket.
        \vs

    \end{parts}

  \hrule

  \question 
    Newtons Second Law gives us the equation 
    $F_{NET}=ma$.  A bicycle has a mass of 120 kg.  It 
    accelerates at a rate of 
    \SI{1.2}{\meter\per\second^2}.  

    \begin{parts}
      
      
      \part
        Calculate the net force acting on the bicycle.
        \vs

      \part 
        The bicycle experiences 52 N of friction.  
        
        Draw a free body diagram.  
        
        {\bf Ignore Air Resistance.}
        

        \begin{center}
          \begin{tikzpicture}
            \draw[pattern=north east lines] 
              (-6,0) rectangle (7,-.2);
            \path (-1,.75) pic[scale=0.7] {bike};
            \draw[ultra thin, draw=gray!50] 
              (-2.5,1.5) -- + (-1.5,0) 
              ++ (-.5,-.3) -- + (-1.25,0) 
              ++ (-.1,-.3) -- + (-.8,0);
          \end{tikzpicture}
        \end{center}
      
      \part
        Calculate the applied force on the bicycle.
        \vs

    \end{parts}
    
  
  
  
  
  
  
  
  

\end{questions}

\end{document}