\documentclass[12pt]{exam}
\usepackage[phy]{template-for-exam}
\usepackage{tikz,ifthen,multicol,siunitx}
\footer{}{}{}
\header{}{}{}
\shadedsolutions
%\printanswers
\usetikzlibrary{shadings,decorations.pathmorphing,arrows.meta,patterns}
\def\mystrut{\protect\rule[-2.2ex]{0ex}{2.2ex}} 
\qformat{ \textbf{Task \#\thequestion}
  \ifthenelse{\equal{\thequestion}{\thequestiontitle}}
    {}
    {: \emph{\thequestiontitle}}
  \mystrut  \hfill}

\begin{document}

\Large


\begin{questions}

  %Same as Problem #1 on Light #2

\question
  A \emph{diverging} lens has a focal length of 9.0~cm, and an object is placed 3.0~cm from the lens.

  \begin{parts}
    \part What would be the distance of the image from the lens?
    \part What is the magnification of the image?
    \part Will the image be real or virtual?  How do you know?
    \part Will the image be upright or inverted?  How do you know?
  \end{parts}

\vs \hrule \vs


% not the same as Light #2

\question
  A projector LCD screen of height 0.5~cm is placed 1.2~cm away from a lens.  An inverted real image is projected 240~cm away.  

  \begin{parts}
    \part What would be the height of the image?
    \part What is the magnification of the image?
    \part What is the focal length of the lens?
  \end{parts}


\vs \hrule \vs


\question
  A \emph{converging} lens has a focal length of 5~cm.  What is the magnification of an object that is placed 3~cm in front of the lens?


\end{questions}

\end{document}