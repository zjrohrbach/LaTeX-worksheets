\documentclass[10pt]{exam}
\usepackage[phy]{template-for-exam}

\title{Light \#2}
\author{Rohrbach}
\date{\today}

\begin{document}
\maketitle

\begin{questions}

\question
  A \emph{diverging} lens has a focal length of 9.0~cm, and an object is placed 3.0~cm from the lens.

  \begin{parts}
    \part What would be the distance of the image from the lens? \vs
    \part What is the magnification of the image? \vs
    \part Will the image be real or virtual?  How do you know? \vspace{7em}
  \end{parts}


\question
  A rutabaga, which has a height of 44~cm is placed 10~cm in front of a lens.  The image produced has a height of 66~cm and is \emph{inverted}.

  \begin{parts}
    \part What would be the distance of the image from the lens? \vs
    \part What is the magnification of the image? (\emph{Be careful with signs!}) \vs
    \part What is the focal length of the lens? \vs
  \end{parts}

\pagebreak


\question
  A \emph{converging} lens has a focal length of 20.0~m.  It is 60.0~m away from an antelope.

  \begin{parts}
    \part What would be the distance of the image from the lens? \vs
    \part What is the magnification of the image? \vs
  \end{parts}


\question
  You place a penny under a converging lens of focal length 75~mm.  You hold the lens 50~mm away from the penny.  What is the magnification of the lens? 


  \vs[3]
  

\end{questions}

\end{document}