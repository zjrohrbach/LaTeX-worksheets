\documentclass[10pt]{exam}
\usepackage[phy]{template-for-exam}

\title{Measuring Spring Constant}
\author{Rohrbach}
\date{\today}

\begin{document}
\maketitle

\section*{Ground Rules}
\begin{itemize}
  \item NEVER stretch the spring out with your hands
  \item NEVER place more than 200 grams on the spring
  \item When you are not actively taking measurements, do not leave masses hanging on the spring.
  
\end{itemize}

\section*{Spring Constants}

\begin{questions}

\question
  Without anything attached to the spring, measure the spring's equilibrium position off the ground.  Then, place 150 grams on each spring and measure its position off the ground.  Calculate the displacement (difference between the two positions) and convert your answer to meters.  Remember, $\SI{100}{\centi\meter}=\SI{1}{\meter}$.
  %
  \begin{center}
    \renewcommand{\arraystretch}{2}
    \begin{tabular}{|c|c|c|c|c|}
      \hline
      Spring & 
      Equilibrium position &
      Stretched position & 
      Displacement &
      Displacement \\[-1em]
      & (cm) & (cm) & (cm) & (m)
      \\ \hline
      Wide   & & & & \\ \hline
      Narrow & & & & \\ \hline
    \end{tabular}
  \end{center}

\question
  Find the mass in kg (Remember, $\SI{1}{\kilo\gram}=\SI{1000}{\gram}$):  \fillin[][8em]

\question
  Use the equations $F_S=-kd$ and $F_G=mg$ to calculate the spring constant of each spring.  Make sure to use MKS units!

  \begin{tabular}{p{2.8in}|p{2.8in}} 
    \hline
    Wide Spring & Narrow Spring \\\hline
    &\\[15em]
  \end{tabular}


\question
  Think about your numbers.  Does the higher spring constant correspond to the stronger spring or the weaker spring?
  \vs 

\question
  The units for spring constant are N/m.  Explain why this makes sense.
  \vs 
  
\end{questions}
\end{document}