\documentclass[10pt]{exam}
\usepackage[phy]{template-for-exam}
\usepackage{tikz,my-tikz-clipart}
\usetikzlibrary{shadings,decorations.pathmorphing,arrows.meta}
%\printanswers
\shadedsolutions

\title{Unit 08 Review (Simle Harmonic Motion \& Waves)}
\author{Rohrbach}
\date{\today}

\newcommand{\printeqs}{
  \ifprintanswers
  \else
    \begin{center}
      \begin{tabular}{|*{13}{c}|}
        \hline 
        &&&&&&&&&&&&\\
        &
        $T_P=2\pi\sqrt{\frac{\ell}{g}}$ &&
        $T_S=2\pi\sqrt{\frac{m}{k}}$ &&
        $F_P=-mg\theta$ &&
        $F_S=-kd$ &&
        $F_G=mg$ &&
        $v=f\lambda$&\\
        &&&&&&&&&&&&\\
        \hline
      \end{tabular}
    \end{center}
  \fi
}


\begin{document}
\maketitle

\printeqs

\begin{questions}

\question 
  You are floating in the ocean.  The waves have an amplitude of 1.5 meters.  The frequency with which you bob up and down is 0.2 Hz.  How far apart are the waves if they are traveling at 3 m/s?
  
  \begin{solution}[\stretch{1}]
    Knowns/Unknowns: $A=1.5$~m, $f=0.2$~Hz, $v=3$~m/s.

    \begin{align*}
      v &= f\lambda \\
      3 &= (0.2)\lambda \\
      \SI{15}{\meter} &= \lambda
    \end{align*}

  \end{solution}

\question
  A 0.3-kg mass is attached to a vertical spring. When the mass is attached, the spring stretches by 0.15~m. Calculate the spring constant of the spring.

  \begin{solution}[\stretch{1}]
    Knowns/Unknowns: $m=0.3$~kg, $d=-0.15$~m, $k=$~?.

    Since the spring is being stretched by gravity, $F_S=F_G$.  Therefore,

    \begin{align*}
      F_S &= F_G \\
      -kd &= mg  \\
      -k(-0.15) &= (0.3)(9.8) \\
      k &= \SI{19.6}{\newton\per\meter}
    \end{align*}

  \end{solution}

\question
  Calculate the period and frequency of a pendulum with length 1.4~m.

  \begin{solution}[\stretch{1}]
    Knowns/Unknowns: $\ell=1.4$~m, $T=$~?, $f=$~?.

    \begin{align*}
      T_P &= 2\pi \sqrt{\frac{\ell}{g}} 
            = 2(3.14) \sqrt{\frac{1.4}{9.8}} 
            = \SI{2.37}{\second} \\\\
      f   &= \frac{1}{T} = \frac{1}{2.37} = \SI{0.42}{\hertz}
    \end{align*}

  \end{solution}

\question
  A spring makes 9 oscillations in 15 s. The spring constant is 80~N/m.  What mass is on the spring?
  
  \begin{solution}[\stretch{1}]
    Knowns/Unknowns: $\#osc=9$, $t=15$~s, $k=80$~N/m $m=$~?.
    
    First, $T = \frac{15}{9} = \SI{1.67}{\second}$.  Then,
    
    \begin{align*} 
      T_S  &= 2\pi \sqrt{\frac{m}{k}} \\
      1.67 &= 6.28 \sqrt{\frac{m}{80}}\\
      2.78 &= \frac{39.48m}{80} \\
      \SI{5.67}{\kilo\gram} &= m
    \end{align*}
  \end{solution}

\ifprintanswers
\else
  \begin{EnvUplevel}
    \noindent
    {\small Answers: \hspace{\stretch{1}}
      (1) $\lambda=\SI{15}{\meter}$ 
                                        \hspace{\stretch{1}}
      (2) $k=\SI{19.6}{\newton\per\meter}$ 
                                        \hspace{\stretch{1}}
      (3) $T=\SI{2.37}{\second}$; $f=\SI{0.42}{\hertz}$
                                        \hspace{\stretch{1}}
      (4) $m=\SI{5.67}{\kilo\gram}$
    }
  \end{EnvUplevel}
\fi

\pagebreak

\question
  What are the four equations you need to have memorized?

  \begin{solution}[\stretch{2}]
    \begin{align*}
      T &= \frac{t}{\#osc} &
      f &= \frac{\#osc}{t} &
      T &= \frac{1}{f} &
      f &= \frac{1}{T} &
    \end{align*}
  \end{solution}

\question
  Define the following:

  \begin{parts}
    \part amplitude
    
      \begin{solution}[\stretch{1}]
        the maximum displacement from equilibrium
      \end{solution}

    \part equilibrium

      \begin{solution}[\stretch{1}]
        the point where restoring force is zero
      \end{solution}

    \part frequency

      \begin{solution}[\stretch{1}]
        how many oscillations happen in a second
      \end{solution}

    \part longitudinal wave

      \begin{solution}[\stretch{1}]
        a wave in which the particles in the medium move parallel to the motion of the wave
      \end{solution}

    \part medium

      \begin{solution}[\stretch{1}]
        the matter that waves travel through
      \end{solution}

    \part period

      \begin{solution}[\stretch{1}]
        the time of one oscillation
      \end{solution}

    \part restoring force

      \begin{solution}[\stretch{1}]
        a force that pulls an object toward a fixed equilibrium point
      \end{solution}

    \part spring constant

      \begin{solution}[\stretch{1}]
        $k$, a measure of the stregth of the spring.  Measured in N/m.
      \end{solution}

    \part transverse wave

      \begin{solution}[\stretch{1}]
        a wave in which the particles in the medium move perpendicular to the motion of the wave
      \end{solution}

    \part wavelength

      \begin{solution}[\stretch{1}]
        the distance before a wave repeats itself
      \end{solution}

  \end{parts}

\question 
  Explain why an oscillator keeps moving when it gets to equilibrium, even though the net force there is zero.

  \begin{solution}[\stretch{2}]
    the inertia carries it past
  \end{solution}

\question
  Why doesn't amplitude affect period?

  \begin{solution}[\stretch{2}]
    Oscillators with a larger amplitude have a larger speed.  They also travel a farther distance.  These two effect counteract each other
  \end{solution}

\question 
  What two factors affect the period of a spring?  What two factors affect the period of a pendulum?

  \begin{solution}[\stretch{2}]
    Mass and spring constant affect the period of a spring.  Length and acceleration of gravity affect the period of a pendulum.
  \end{solution}

\question
  What do waves transport and what do they not transport?

  \begin{solution}[\stretch{2}]
    a wave transports energy without transporting matter
  \end{solution}

\question
  Draw an example of constructive and destructive intereference.  Label each.

  \begin{solution}[\stretch{2}]
    
  \end{solution}

\question
  What is the only thing you can do to change the speed of a wave?
  
  \begin{solution}[\stretch{2}]
    the only thing that affects the speed of a wave is the medium it is traveling through
  \end{solution}

\question
  How are frequency and wavelength related?

  \begin{solution}[\stretch{2}]
    they are inversely proportional to each other
  \end{solution}

\end{questions}

\end{document}