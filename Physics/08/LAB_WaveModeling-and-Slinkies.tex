\documentclass[10pt]{exam}
\usepackage[phy]{template-for-exam}
\usepackage{silence}
\usepackage{tikz}
\WarningFilter{latex}{Label `question}
\WarningFilter{latex}{There were multiply-defined labels}
\usepackage{pgfplots}
\pgfplotsset{
    compat=1.18,
    pulseplot/.append style={
      xmin =-6.5,
      xmax = 6.5,
      ymin =-2.5,
      ymax = 2.5,
      axis lines = none,
      xtick = {},
      ytick = {},
      height = 5cm,
      width = 7.5cm,
      grid = major,
      grid style = {thick, dotted},
      tick label style = {font=\small}
    },
  }
\pgfmathdeclarefunction{gauss}{2}{%
  \pgfmathparse{1/(#2*sqrt(2*pi))*exp(-((x-#1)^2)/(2*#2^2))}%
}

\title{Wave Modeling \& Slinky Labs}
\author{Rohrbach}
\date{\today}

\begin{document}
\maketitle

\vspace{-1em}


\section{Wave Modeling}

A transverse wave may be modeled when the marker moves back and forth perpendicularly to the direction of the moving paper.  If done carefully, you should see a nice consistent wave shape.

\subsection{Data}

Place the large sheet of butcher paper on the table.  One group member should steadily pull the paper across the table, while another member moves the marker back and forth over the paper.  The arm should swing freely like a pendulum.  One group member should use a stopwatch to measure the amount of time needed to sketch the wave motion across the entire length of the paper.

\vspace{1em}

\noindent
Do your best to draw an equilibrium line down through the center of your wave.

\vspace{2em}

\renewcommand{\arraystretch}{1.5}

\begin{tabular}{p{.7\textwidth}p{.2\textwidth}}
  \hline
  Record the total time ($t$) of the wave motion. \\
  & $t=$ \\\hline
  Measure the total distance ($d$) that your wave traveled. \\
  & $d=$ \\\hline
  Record the total number of wavelengths ($\#osc$) for your wave (remember, a wavelength is measured from crest to crest and it is ok to have half of a wave). \\
  & $\#osc=$ \\\hline
  Determine your wave's period. \\
  & $T=$ \\\hline
  Use your period to find the frequency of the wave. \\
  & $f=$ \\\hline
  Label one full wavelength ($\lambda$) on your wave (measure and include its value).  Measure the rest of your wavelengths, and calculate the average wavelength for your wave. \\
  & $\lambda=$ \\\hline
  Label the amplitude ($A$) of one of the wave peaks (measure and include its value).  Measure the rest of your amplitudes (both crests and troughs) and calculate the average amplitude for your wave. \\
  & $A=$ \\\hline
\end{tabular}

\pagebreak

\begin{questions}
  \uplevel{
    \subsection{Calculations}
  }

  \question
    Calculate the velocity of the wave using $v=d/t$. \vs 
  \question 
    Calculate the velocity of the wave using $v=f\lambda$. \vs 
  \question 
    Calculate the percent error using these two calculations.  Use \#1 as your expected value.
    \begin{align*}
      \text{\% error} &=  
      \frac{
        \left|\text{measured}-\text{expected}\right|
        }{
          \text{expected}
        } 
      \times 100
    \end{align*}
    \vs


    \uplevel{
      \subsection{Conclusion Questions}
    } 

    \question
      How close were your two velocity calculations?  What could have caused the error?
      \vs 

    \question
      How well do you think your wave turned out?  What could you do to get a better wave next time?
      \vs
    
    \question
      What would you have to do in order to increase the frequency of the wave?
      \vs 

    \question
      What do you suppose would happen to the wavelength when you increased the frequency?
      \vs

\end{questions}

\pagebreak

\begin{questions}

  \uplevel{
    \section{Slinkies}
    \subsection{Pulses}
  }
  
  \question
    Send a {\bf transverse} pulse (one at a time, and just one person) down a stretched slinky.  Make sure to make them side to side. Observe the motion of the spring's coils. How are they moving?
    \vs 
  
  \question
    Send a large pulse and a small pulse down the slinky. Which one appears to travel faster? 
    \vs
  
  
  
  \question
    Change the tension of the spring by coming closer together or moving further apart.  How (if at all) does tension in the slinky seem to affect the speed of the wave?
    \vs
  
  \question
    Send another transverse pulse.  What happens to the orientation of the wave (right side up or upside down) when it reaches the other side of the slinky and bounces back?
    \vs
  
  \question
    Stretch the large spring and send a {\bf longitudinal} pulse down the spring.  Observe the motion of the spring coils.  How are they moving?
    \vs
  
  
  
  \uplevel{
    \subsection{Wave Interference}
  }
  
  \question
    Have each person send a {\bf singular wave pulse} down the spring at the same time.  Make sure the amplitudes of the waves are on the {\bf same} side of the spring.  
  
    \begin{parts}
      \part 
        Do the waves bounce off each other or pass through each other?

        \begin{flushright}
          \begin{tikzpicture}
            \begin{axis}[pulseplot]
              \addplot[smooth,domain=-6:6,samples=50,ultra thick] {1.3*(gauss(-4,.5)+gauss(4,.5))};
              \draw[->,thick] (-3,.7) -- (-2,.7);
              \draw[->,thick] (3,.7) -- (2,.7);
            \end{axis}
          \end{tikzpicture}
        \end{flushright}


      \part 
        What do you observe where the waves meet (in the middle)? \vs
      \part 
        Which kind of interference is this? \vs 
    \end{parts}
  
  \pagebreak
  
  \question
    Have each person send a singular wave pulse down the spring at the same time.  Make sure the amplitudes of the waves are on the {\bf opposite} sides of the spring.  What do you notice?
  
    \begin{parts}
      \part 
        Do you think the waves bounce off each other or pass right through?

        \begin{flushright}
          \begin{tikzpicture}
            \begin{axis}[pulseplot]
              \addplot[smooth,domain=-6:6,samples=50,ultra thick] {1.3*(gauss(-4,.5)-gauss(4,.5))};
              \draw[->,thick] (-3,.7) -- (-2,.7);
              \draw[->,thick] (3,-.7) -- (2,-.7);
            \end{axis}
          \end{tikzpicture}
        \end{flushright}

      \part 
        What do you observe where the waves meet (in the middle)? \vs
      \part 
        Which kind of interference is this? \vs 
    \end{parts}



  \uplevel{
    \subsection{Standing Waves}
  }
    
  \question
    While holding one end of the large spring firmly in place, move the other end of the spring continuously back and forth to send a continuous wave down the spring.  Adjust your frequency until a standing wave with three ``loops'' is obtained (it should look like a still wave that keeps flipping upside-down).  Draw a sketch below.  \emph{ Note: ``STANDING WAVE'' IS THE NAME OF THE WAVE, NOT WHAT YOU ARE SUPPOSED TO DO TO MAKE THEM.  THEY SHOULD STILL MOVE SIDE TO SIDE AND NOT UP AND DOWN!!!}
  
    \begin{tikzpicture}
      \draw[dotted] (0,0) grid[step=0.5cm] (15,3);
    \end{tikzpicture}
  
  
  \question
    Now change your frequency of vibration until more loops are formed (shake the spring faster).  What happens to the wavelength?
    \vs
  
  
  \question
    Try and make standing waves with four or more loops, and sketch your waves below.
  
    \begin{tikzpicture}
      \draw[dotted] (0,0) grid[step=0.5cm] (15,3);
    \end{tikzpicture}


\end{questions}






\end{document}