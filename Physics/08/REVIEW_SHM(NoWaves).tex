\documentclass[10pt]{exam}
\usepackage[phy]{template-for-exam}
\usepackage{tikz,my-tikz-clipart}
\usetikzlibrary{shadings,decorations.pathmorphing,arrows.meta}
%\printanswers
\shadedsolutions

\title{Unit 08(A) Review (Simle Harmonic Motion)}
\author{Rohrbach}
\date{\today}

\newcommand{\printeqs}{
  \ifprintanswers
  \else
    \begin{center}
      \begin{tabular}{|*{11}{c}|}
        \hline 
        &&&&&&&&&&\\
        &
        $T_P=2\pi\sqrt{\frac{\ell}{g}}$ &&
        $T_S=2\pi\sqrt{\frac{m}{k}}$ &&
        $F_P=-mg\theta$ &&
        $F_S=-kd$ &&
        $F_G=mg$ &\\
        &&&&&&&&&&\\
        \hline
      \end{tabular}
    \end{center}
  \fi
}


\begin{document}
\maketitle

\printeqs

\begin{questions}


\question
  A 0.3-kg mass is attached to a vertical spring. When the mass is attached, the spring stretches by 0.15~m. Calculate the spring constant of the spring.

  \begin{solution}[\stretch{1}]
    Knowns/Unknowns: $m=0.3$~kg, $d=-0.15$~m, $k=$~?.

    Since the spring is being stretched by gravity, $F_S=F_G$.  Therefore,

    \begin{align*}
      F_S &= F_G \\
      -kd &= mg  \\
      -k(-0.15) &= (0.3)(9.8) \\
      k &= \SI{19.6}{\newton\per\meter}
    \end{align*}

  \end{solution}

\question
  Calculate the period and frequency of a pendulum with length 1.4~m.

  \begin{solution}[\stretch{1}]
    Knowns/Unknowns: $\ell=1.4$~m, $T=$~?, $f=$~?.

    \begin{align*}
      T_P &= 2\pi \sqrt{\frac{\ell}{g}} 
            = 2(3.14) \sqrt{\frac{1.4}{9.8}} 
            = \SI{2.37}{\second} \\\\
      f   &= \frac{1}{T} = \frac{1}{2.37} = \SI{0.42}{\hertz}
    \end{align*}

  \end{solution}

\question
  A spring makes 9 oscillations in 15 s. The spring constant is 80~N/m.  What mass is on the spring?
  
  \begin{solution}[\stretch{1}]
    Knowns/Unknowns: $\#osc=9$, $t=15$~s, $k=80$~N/m $m=$~?.
    
    First, $T = \frac{15}{9} = \SI{1.67}{\second}$.  Then,
    
    \begin{align*} 
      T_S  &= 2\pi \sqrt{\frac{m}{k}} \\
      1.67 &= 6.28 \sqrt{\frac{m}{80}}\\
      2.78 &= \frac{39.48m}{80} \\
      \SI{5.67}{\kilo\gram} &= m
    \end{align*}
  \end{solution}

\ifprintanswers
\else
  \begin{EnvUplevel}
    \noindent
    {\small Answers: \hspace{\stretch{1}}
      (1) $k=\SI{19.6}{\newton\per\meter}$ 
                                        \hspace{\stretch{1}}
      (2) $T=\SI{2.37}{\second}$; $f=\SI{0.42}{\hertz}$
                                        \hspace{\stretch{1}}
      (3) $m=\SI{5.67}{\kilo\gram}$
    }
  \end{EnvUplevel}
\fi

\pagebreak

\question
  What are the four equations you need to have memorized?

  \begin{solution}[\stretch{2}]
    \begin{align*}
      T &= \frac{t}{\#osc} &
      f &= \frac{\#osc}{t} &
      T &= \frac{1}{f} &
      f &= \frac{1}{T} &
    \end{align*}
  \end{solution}

\question
  Define the following:

  \begin{parts}
    \part amplitude
    
      \begin{solution}[\stretch{1}]
        the maximum displacement from equilibrium
      \end{solution}

    \part equilibrium

      \begin{solution}[\stretch{1}]
        the point where restoring force is zero
      \end{solution}

    \part frequency

      \begin{solution}[\stretch{1}]
        how many oscillations happen in a second
      \end{solution}

    \part period

      \begin{solution}[\stretch{1}]
        the time of one oscillation
      \end{solution}

    \part restoring force

      \begin{solution}[\stretch{1}]
        a force that pulls an object toward a fixed equilibrium point
      \end{solution}

    \part spring constant

      \begin{solution}[\stretch{1}]
        $k$, a measure of the stregth of the spring.  Measured in N/m.
      \end{solution}

  \end{parts}

\question 
  Explain why an oscillator keeps moving when it gets to equilibrium, even though the net force there is zero.

  \begin{solution}[\stretch{2}]
    the inertia carries it past
  \end{solution}

\question
  Why doesn't amplitude affect period?

  \begin{solution}[\stretch{2}]
    Oscillators with a larger amplitude have a larger speed.  They also travel a farther distance.  These two effect counteract each other
  \end{solution}

\question 
  What two factors affect the period of a spring?  What two factors affect the period of a pendulum?

  \begin{solution}[\stretch{2}]
    Mass and spring constant affect the period of a spring.  Length and acceleration of gravity affect the period of a pendulum.
  \end{solution}

\end{questions}

\end{document}