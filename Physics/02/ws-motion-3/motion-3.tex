\documentclass[12pt]{exam}
\usepackage[phy]{template-for-exam}
\usepackage{empheq}

\newcommand{\printeqs}{
  \begin{center}
    \vspace{-1cm}
    \begin{empheq}[box=\fbox]{align*}
     && v &= \frac{d}{t}    &    a &= \frac{v_f-v_i}{t} &&
    \end{empheq}
  \end{center}
}

\title{Motion \#3}
\author{Rohrbach}


\begin{document}
\maketitle

\printeqs

\noindent
{\bf Please use the proper problem-solving method.}  If you 
get stuck, try your best.  I will be lenient on giving you
credit if you have at least (a) drawn a picture, 
(b) done knowns/unknowns, and (c) chosen an equation!!!


\begin{questions}

\question
  What is the velocity of an ATV that travels 13 m in 2 s?
  \vspace{\stretch{1}}

\question
  How long does it take for a car that is traveling 
  35 m/s to drive {\bf 23 km}?
  \vspace{\stretch{1}}

\question
  How far can a person go if they run at a velocity of 
  8 m/s for {\bf 3 minutes}?
  \vspace{\stretch{1}}

\question
  What is the acceleration of a unicycle that goes from 
  3 m/s to 5 m/s in 7 s?
  \vspace{\stretch{1}}

\pagebreak

\printeqs


\question
  What is the final velocity of a roller coaster 
  that starts at rest and accelerates at a rate of 
  \SI{5}{\meter\per\second^2} for 0.8 s?
  \vspace{\stretch{1}}


\question
  How much time would it take an object accelerating 
  at \SI{9}{\meter\per\second^2} to go from 25 m/s to 56 m/s?
  \vspace{\stretch{1}}


\question
  After accelerating at a rate of \SI{-5}{\meter\per\second^2}
  for 8 seconds, you are now traveling at 11.3 m/s.  How fast 
  were you going before you started accelerating?
  \vspace{\stretch{1}}



  
\end{questions}
\end{document}