\documentclass[11pt]{exam}
\usepackage{unit2}

\title{Kinematics \#2}
\author{Rohrbach}
\date{\today}

\begin{document}
\maketitle

\printeqs

\begin{questions}
  \question  
    It takes a train quite a bit of time to get up to speed.  If it starts at rest and accelerates for \textbf{2.6 km} over a course of \textbf{2 min}, what is its acceleration?
    \vs

  \question
    A truck slams on the brakes to come to a stop before hitting a deer.  The truck accelerates at $-12.9$ m/s$^2$.  If it was originally traveling at 35 m/s before hitting the brakes, how far would it go before it stopped?
    \vs

  \question
    Your car has an acceleration of 3.2 m/s$^2$.  You step on the accelerator to get up to speed as you're merging onto the interstate.  If it takes you 4.1 seconds to get up to a speed of 45 m/s, how fast were you going before you started accelerating?
    \vs

    
  
\end{questions}

\end{document}
