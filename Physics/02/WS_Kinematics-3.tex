\documentclass[10pt]{exam}
\usepackage{unit2}

\title{Kinematics \#3}
\author{Rohrbach}
\date{\today}

\begin{document}
\maketitle

\printeqs

\begin{questions}
  \question  
    The Eiffel Tower is 324 meters tall.  If we make the (totally unrealistic) assumption that there is no air resistance, how fast would a penny dropped from the Eiffel Tower be travelling the moment before it hit the ground?
    \vspace{5cm}

  \question
    You have a kitten named Mittens.  You toss Mittens into the air at an initial velocity of 8 m/s.  
    
    \begin{parts}
      \part
        How high will Mittens go before coming back down (at which point, you, of course, will gently catch her.)
        \vs[3]
      
      \part
        How long does it take Mittens to reach her maximum height?
        \vs[2]
      
      \part
        What is the total time that Mittens is in the air?
        \vs
      
      \part
        What is Mittens' velocity right before she is caught?
        \vs

    \end{parts}
    
  \pagebreak

  \printeqs

  \question
    You stand at the top of a 30-meter cliff and shoot an arrow straight up into the air at an initial velocity of 12 m/s.  You let it fall down to the bottom of the cliff

    \begin{parts}
      \part
        How high above where you shot it will the arrow go?
        \vspace{5cm}

      \part
        How fast will the arrow be travelling when it makes it to the bottom of the cliff?
        \vspace{4cm}

    \end{parts}



  \question
    A stone is hurled straight upward at a speed of 30 m/s from ground level. 

    \begin{parts}
      \part
        What is the total time of flight?
        \vspace{4cm}
      
      \part
        What is the speed at the top of the flight?
        \vs

      \part
        What is the acceleration at the top of the flight?
        \vs
    \end{parts}
    
  
\end{questions}

\end{document}
