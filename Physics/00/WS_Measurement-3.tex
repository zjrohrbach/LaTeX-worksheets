\documentclass[11pt]{exam}
\usepackage[phy]{template-for-exam}
\usepackage{wrapfig}
\usepackage{multicol}

\title{Measurement \#3 \it (Review)}
\author{Rohrbach}
\date{\today}

\begin{document}
\maketitle


\begin{center}
  \begin{tabular}{ccc}
    \hline
    \multicolumn{3}{c}{Metric Prefixes}                 \\
    \hline
    \SI{}{\kilo\relax}  & kilo-            & \SI{e3}{}  \\
           --           & (\textit{base})  & $10^0$     \\
    \SI{}{\centi\relax} & centi-           & \SI{e-2}{} \\
    \SI{}{\milli\relax} & milli-           & \SI{e-3}{} \\
    \SI{}{\micro\relax} & micro-           & \SI{e-6}{} \\
    \SI{}{\nano\relax}  & nano-            & \SI{e-9}{} \\
    \hline
  \end{tabular}
\end{center}


\begin{questions}


  
  \question Complete the following unit conversions.

  \begin{multicols}{2}
    \begin{parts}
      \part 
        $\SI{2500}{\micro\meter}=$ ? \SI{}{\meter} 
        \vspace{3em}

      \part 
        $\SI{326000}{\milli\gram}=$ ? \SI{}{\kilo\gram} 
        \vspace{3em}
      
      \part 
        $\SI{4.8}{\meter}=$ ? \SI{}{\milli\meter} 
        \vspace{3em}

      \part 
        $\SI{2.1}{\second}=$ ? \SI{}{\milli\second} 
        \vspace{3em}

    \end{parts}
  \end{multicols}

  \question Express each of these measurements in MKS units:

  \begin{multicols}{2}
    \begin{parts}
      \part \SI{9.1}{\kilo\meter} \vspace{2em}
      \part \SI{53}{\centi\meter} \vspace{2em}
      \part \SI{320}{\gram} \vspace{2em}
      \part \SI{1.2}{\hour} \vspace{2em}
    \end{parts}
  \end{multicols}

  \question Express these numbers in scientific notation.

  \begin{multicols}{2}
    \begin{parts}
      \part \SI{0.025}{} \vspace{2em}
      \part \SI{1150000}{} \vspace{2em}
      \part \SI{0.0000771}{} \vspace{2em}
      \part \SI{6070}{} \vspace{2em}
    \end{parts}
  \end{multicols}

  \question Express these numbers in standard form.

  \begin{multicols}{2}
    \begin{parts}
      \part \SI{2.96e7}{} \vspace{2em}
      \part \SI{6.02e-3}{} \vspace{2em}
      \part \SI{6.67e-11}{} \vspace{2em}
      \part \SI{9.8e5}{} \vspace{2em}
    \end{parts}
  \end{multicols}


  \pagebreak


  \question 
    Use your calculator to perform the following calculations:

  \begin{multicols}{2}
    \begin{parts}
      \part
        $
        \left( \SI{5.95e15}{} \right) \div 
        \left( \SI{7.35e-20}{} \right) =
        $
        \vspace{4em}

      \part
        $
        \left( \SI{1.23e9}{} \right) \times
        \left( \SI{4.23e-8}{} \right) =
        $
        \vspace{4em}

    \end{parts}
  \end{multicols}

  \question
    You perform an experiment to measure the density of
    aluminum.  After performing five trials, you get the
    following results:
    %
    \begin{center}
      \begin{tabular}{cc}
        \hline
        \bf Trial & {\bf Result} (g/mL) \\
        \hline
          1       &   2.5               \\
          2       &   3.2               \\
          3       &   2.9               \\
          4       &   3.0               \\
          5       &   2.6               \\
        \hline \hline
      \end{tabular}

      \begin{parts}

        \part 
          Are your measurements precise?  Explain. \vs

        \part 
          The widely accepted value for the density of
          aluminum is \SI{2.7}{\gram\per\milli\liter}. Are 
          your measurements accurate?  Explain. \vs

        \part
          Calculate the percent error based upon your 
          average measurement.  Is your percent error 
          reasonable?  Explain. \vs

      \end{parts}
    \end{center}
  
\end{questions}


\end{document}