\documentclass[10pt]{exam}
\usepackage[phy]{template-for-exam}

\title{Mini-Lab - \\ Accuracy and Precision}
\author{Rohrbach}


\begin{document}

\maketitle

\begin{questions}

  \question
    For reasons we will discuss in the coming units, the 
    time it takes for a ball to drop from a given height is
    given by the equation
    %
    \begin{equation*}
      t = \sqrt{\frac{h}{4.9}}\,\, .
    \end{equation*}
    %
    Measure the height from which the ball is dropped.  Use the equation to calculate the time it shoult take for the ball to hit the ground.

    \vspace{\stretch{1}}


  \question
    Is the answer you got the expected or the measured value? How do you know?

    \vspace{\stretch{1}}

  \question
    We will now measure the time of drop several times as a class, write down their measurements below.
    %
    \begin{center}
    \begin{tabular}[c]
      {|m{10em}|w{c}{5em}|w{c}{5em}|w{c}{5em}|}
      \hline
      Timer: & Trial \#1 & Trial \#2 & Trial \#3 \\
      \hline
             &           &           &           \\[2em]
      \hline
             &           &           &           \\[2em]
      \hline
             &           &           &           \\[2em]
      \hline
    \end{tabular}
    \end{center}

  \question
    Were the times accurate?  Were they precise?
    Explain.

    \vspace{\stretch{1}}

  \question
    Calculate the percent error.

    \vspace{\stretch{1}}

  \question
    After having calculated the percent error, make 
    a further comment about the accuracy of the
    measurements.

    \vspace{\stretch{1}}

\end{questions}


\end{document}