\documentclass[10pt]{exam}
\usepackage[phy]{template-for-exam}
\usepackage{tikz}
\usetikzlibrary{decorations.pathmorphing,patterns}
\header{Name:}{Date:}{Period: \hspace{2cm} ID:A}


\title{Projectile Motion - Quiz II}
\author{Rohrbach}
\date{\today}

\begin{document}
\maketitle

\noindent
An arrow is shot with a velocity of 120 m/s at a $72^\circ$ angle.  

\vspace{2em}


\begin{parts}
  \part
    Label this diagram of the situation and {\bf make a T-chart} to list your knowns and unknowns. (3 points)

    \vspace{1em}

    \begin{tikzpicture}
      \draw[dashed,red] (2.5,2) parabola (0,0);
      \draw[dashed,red] (2.5,2) parabola (5,0);
      \draw (-1,0) -- ++(7,0);
      \draw[ultra thick, blue, ->] (0,0) -- ++(65:.6);
      \draw[ultra thick, blue, ->] (0,0) -- ++(65:.6);
      \draw[ultra thick, blue, ->] (2.3,2) -- ++(0:.6);
      \draw[ultra thick, blue, ->] (5,0) ++(125:.6)  
        -- ++(-65:.6);
    \end{tikzpicture}

    \vspace{1em}

  \part
    Draw a triangle and calculate the $x$- and $y$- velocities of the arrow at the moment it is shot into the air. (3 points)
    \vs
  
  \part
    What is the maximum height that the arrow will reach? (3 points)
    \vs
  
  \part
    How long will the arrow be in the air? (3 points)
    \vs
  
  \part
    How far forward will the arrow go before hitting the ground? (3 points)
    \vs

    
\end{parts}


\end{document}