\documentclass[12pt]{exam}
\usepackage[phy]{template-for-exam}
\usepackage{tikz,ifthen,multicol}
\footer{}{}{}
\header{}{}{}
\shadedsolutions
%\printanswers



\begin{document}

\Large


\vspace*{\stretch{1}}

\newcommand{\theq}{
  \noindent
  During the Civil War, an artillery commander is preparing to fire a cannon positioned on level ground. The cannon is aimed at an angle of $30^\circ$ above the horizontal and fires a cannonball with an initial velocity of 200~m/s.  Calculate the horizontal distance where the cannonball would land.

  \begin{solution}
    \begin{align*}
      v_{0x} &= 200 \cos(30^\circ) = 173.2 \, \text{m/s}\\
      v_{0y} &= 200 \sin(30^\circ) = 100 \, \text{m/s} \\
      t &= \frac{2 v_{0y}}{g} = \frac{2 \cdot 100}{9.81} = 20.39 \, \text{s} \\
      R &= v_{0x} \cdot t = 173.2 \cdot 20.39 = 3525.5 \, \text{m}
    \end{align*}

    Extension A: if there is a hill halfway between you and your target that is 550 meters high, will your cannonball make it over?  (no, the  max height is 509.68~m).

    Extension B: What could you do instead? (Try 60 degrees; same range, max height is 1521.5~m)
\end{solution}
}

\theq

\vs \hrule \vs

\theq

\vs

\pagebreak

\printanswers \theq





\end{document}