\documentclass[10pt]{exam}
\usepackage[phy]{template-for-exam}

\title{Air Rocket Lab}
\author{Rohrbach}
\date{\today}

\begin{document}
\maketitle

\section*{Purpose}

You will have a target located on the football field.  Your goal is to hit the target in three shots of your air rocket.

\begin{center}
  Your group's target: \fillin[][10em]
\end{center}

\section*{Pre-Lab}

\begin{questions}

  \question
    You will be shooting your rocket from a launch pad which can have launch angles from zero degrees (shooting straight across the ground) to ninety degrees (shooting straight up).  Below, draw a general diagram of the motion of your rocket.  Please label the following: initial resultant velocity of the rocket, maximum height, angle, displacement in the $x$-direction.

    \vspace{10em}


  \question
    There are three velocity caps that you can use: low, medium, or high.  As a group, decide which velocity cap you want to start with and why.
    \vs

  \question
    Your first shot may be the most important, since it sets up the rest of your lab.  As a group, decide what angle you want to launch your rocket at.  Explain your reasoning
    \vs

  \question
    Think, if your first launch goes too far, what do you think you should do to the angle of your second shot?
    \vs


  \pagebreak
  \question
    Your instructor will show you how to use your iPhone to measure the launch angle of your rocket.  Take notes below on how to achieve this:
    \vs

  \question
    In order to practice, set up your rocket for the angle of the first shot.  Call your instructor over when you are finished so he can check it.
  
\end{questions}

\section*{Launch Day}

You are only allowed three shots.  Do not waste them!

\paragraph{Trial 1 - the test shot:}  Set up your rocket at the end zone directly across from your target.  In the table below, note your velocity setting (low, medium, or high), angle, and the $x$-displacement your rocket traveled.

\noindent
\begin{tabular}{|*4{l|}}
  \hline
  & velocity cap & angle (degrees) & $x$-displacement\\\hline
  &&& \\
  Trial \#1 &&& \\
  \hline
\end{tabular}

\noindent
Describe what happened. (Did it go too far, too short, too far to the left?)
\vspace{5em}

\noindent
Explain what you need to do differently in trial \#2, what you are going to change, {\bf and why} you are changing that specific thing.  Be thorough and use physics concepts.
\vspace{5em}


\paragraph{Trial 2 - adjustments:}  Make any changes to your set up and note your changes below.

\noindent
\begin{tabular}{|*4{l|}}
  \hline
  & velocity cap & angle (degrees) & $x$-displacement\\\hline
  &&& \\
  Trial \#2 &&& \\
  \hline
\end{tabular}

\noindent
Describe what happened. (Did it go too far, too short, too far to the left?)
\vspace{5em}

\noindent
Explain what you need to do differently in trial \#3, what you are going to change, {\bf and why} you are changing that specific thing.  Be thorough and use physics concepts.
\vspace{5em}

\pagebreak
\paragraph{Trial 3 - make it happen:}  Make any changes to your set up and note your changes below.

\noindent
\begin{tabular}{|*4{l|}}
  \hline
  & velocity cap & angle (degrees) & $x$-displacement\\\hline
  &&& \\
  Trial \#3 &&& \\
  \hline
\end{tabular}

\noindent
Describe what happened. (Did it go too far, too short, too far to the left?)
\vspace{5em}


			
\section*{Conclusions}
Write a few sentences to answer the following questions:

a.	What was the purpose of this lab? 

b.	Were you successful?  Why or why not? 

c.	What errors affected your experiment and what could you do in the future to improve your results?


\end{document}