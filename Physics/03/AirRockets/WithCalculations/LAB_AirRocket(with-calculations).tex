\documentclass[10pt]{exam}
\usepackage[phy]{template-for-exam}

\title{Air Rocket Lab}
\author{Rohrbach}
\date{\today}

\begin{document}
\maketitle


\section*{Grading Rubric}

  \paragraph{Diagram} Question \#\ref{diagram}
  \vspace{0.5em}
  \begin{checkboxes} \small
    \choice (2 points) Complete and correct 
    \choice (1 point) Missing some key parts
    \choice (0 points) Missing/Incomplete
  \end{checkboxes}


  \paragraph{Calculation} Questions \#\ref{t-chart}-\ref{convert}
  \vspace{0.5em}
  \begin{checkboxes} \small
    \choice (5 points) Complete, correct, and easy to follow 
    \choice (3 points) The T-Chart is incomplete or it is difficult to follow what you are doing in the calculation
    \choice (0 points) No work is shown
  \end{checkboxes}


  \paragraph{Data Table} Launches 1-3
  \vspace{0.5em}
  \begin{checkboxes} \small
    \choice (5 points) Discussion of how the launches went are well reasoned and thorough.
    \choice (3 points) Your explanations are missing or incomplete
    \choice (0 points) You didn't do this part
  \end{checkboxes}

  \paragraph{Conclusion} Final Paragraph
  \vspace{0.5em}
  \begin{checkboxes} \small
    \choice (3 points) Conclusion is detailed and complete
    \choice (2 points) There are some minor errors/lack of detail in explaining the purpose or procedure of the lab
    \choice (0 points) Missing/Incomplete
  \end{checkboxes}

  \section*{Purpose}

  Your goal is to predict where your rocket will land.  Your group will be assigned a launch angle between 10 and 80 degrees.

  \begin{center}
    Your group's starting angle: \fillin[][3em]$^\circ$
  \end{center}

\section*{Pre-Lab}

\begin{questions}

  \question \label{diagram}
    You will be shooting your rocket from a launch pad which can have launch angles from zero degrees (shooting straight across the ground) to ninety degrees (shooting straight up).  Below, draw a general diagram of the motion of your rocket.  Please label the following: initial resultant velocity of the rocket, maximum height, angle, displacement in the $x$-direction.

    \vspace{6em}


  \pagebreak

  \question \label{t-chart}
    You will be pressurize your rocket to 70 PSI.  This will give the rocket an initial resultatant velocity of approximately \textbf{22~m/s}.  List all of your known values for this lab.  Think about what values you need from your instructor.  Make your T-chart below:

    \begin{center}
      \begin{tabular}{l|c}
        $x$-direction & $y$-direction \\\hline
        & \\[6em]
      \end{tabular}
    \end{center}

  \question \label{calc}
    Solve for how far away your rocket will land.  A few hints:

    \begin{parts}
      \part First, find the $x$- and $y$- components of the initial velocity
      \part Second, you need to know how long it will be in the air
      \part Finally, you should be able to use these calculations to find the $x$-displacement.
    \end{parts}

    \vs

  \question \label{convert}
    Multiply your answer for $x$-displacement by \textbf{1.093} in order to convert it from meters to yards.
    \vspace{4em}

  \pagebreak

  \question \label{too-far}
    Think, if your first launch goes too far, what do you think you should do to the angle of your second shot?
    \vspace{4em}

  \question \label{too-short}
    Think, if your first launch does not go far enough, what do you think you should do to the angle of your second shot?
    \vspace{4em}


\end{questions}

\section*{Launch Day}

You are only allowed three shots.  Do not waste them!

\paragraph{Trial 1 - the test shot:}  Set up your rocket at the end zone directly across from your target.  In the table below, note your velocity setting (low, medium, or high), angle, and the $x$-displacement your rocket traveled.

\noindent
\begin{tabular}{|*4{l|}}
  \hline
  & pressure (PSI) & angle (degrees) & $x$-displacement\\\hline
  &&& \\
  Trial \#1 &&& \\
  \hline
\end{tabular}

\noindent
Describe what happened. (Did it go too far, too short, too far to the left?)
\vs

\noindent
Explain what you need to do differently in trial \#2, what you are going to change, {\bf and why} you are changing that specific thing.  Be thorough and use physics concepts.
\vs


\paragraph{Trial 2:}  Make any changes to your set up and note your changes below.

\noindent
\begin{tabular}{|*4{l|}}
  \hline
  & pressure (PSI) & angle (degrees) & $x$-displacement\\\hline
  &&& \\
  Trial \#2 &&& \\
  \hline
\end{tabular}

\noindent
Describe what happened. (Did it go too far, too short, too far to the left?)
\vs

\noindent
Explain what you need to do differently in trial \#3, what you are going to change, {\bf and why} you are changing that specific thing.  Be thorough and use physics concepts.
\vs

\pagebreak

\paragraph{Trial 3:}  Make any changes to your set up and note your changes below.

\noindent
\begin{tabular}{|*4{l|}}
  \hline
  & pressure (PSI) & angle (degrees) & $x$-displacement\\\hline
  &&& \\
  Trial \#3 &&& \\
  \hline
\end{tabular}

\noindent
Describe what happened. (Did it go too far, too short, too far to the left?)
\vspace{5em}
			
\section*{Conclusions}
Write a paragraph to answer the following questions:

a.	What was the purpose of this lab? 

b.	Were you successful?  Why or why not? 

c.	What errors affected your experiment and what could you do in the future to improve your results?


\end{document}